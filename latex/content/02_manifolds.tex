\chapter{Manifolds}

Manifolds are the fundamental spaces used in physics. They provide a framework to describe topological spaces that locally resemble Euclidean spaces, allowing for the application of known methods from calculus and linear algebra.

\section{Preliminaries from topology}

Although this thesis will not focus on introducing topology, a few important results will be given here, which are necessary to understand the definition of manifolds. The following definitions and theorems are taken from \cite{NakaharaGeometrytopologyphysics2005}

A \textbf{topological space} is a set \( X \) equipped with a collection of open sets \( \mathcal{T} = \{ U_i \mid i\in I\} \) such that:
\begin{itemize}
    \item \( \emptyset, X \in \mathcal{T} \)
    \item For any subcollection $J$ of $I$ the Union of corresponding open sets is itself an open set \( \bigcup_{j\in J} U_j \in \mathcal{T} \)
    \item For any finite subcollection $K$ of $I$ the intersection of the corresponding open sets is open: \( \cap_{k \in K} U_k \in \mathcal{T} \)
\end{itemize}

A family $\{O_i\}$ of (open) subsets of $X$ is called an (open) covering of $X$ if \( X = \bigcup_i O_i \).

A subset $N$ is called a \textbf{neighborhood} of a point \( p \in X \) if there exists at least one open set \( U \in \mathcal{T} \) such that \( p \in U \subset N \). A topological space is called \textbf{Hausdorff} if for any two distinct points \( p, q \in X \) there exist neighborhoods \( N_p, N_q \) such that \( N_p \cap N_q = \emptyset \).

A map \( f: X \to Y \) between two topological spaces is called \textbf{continuous} if for every open set \( V \subset Y \) the preimage \( f^{-1}(V) \) is an open set in \( X \). If the inverse \( f^{-1} : Y \to X \) is also continuous, then \( f \) is called a \textbf{homeomorphism}. Two topological spaces are called \textbf{homeomorphic} if there exists a homeomorphism between them.

\section{Differentiable Manifolds}

A Hausdorff topological space $(M,\mathcal{T})$ is called a \textbf{d-dimensional manifold} if there exists an open covering \( \{U_i\} \) and a family of homeomorphisms \( \varphi_i: U_i \to \varphi_i(U_i) \subset \mathbb{R}^d \). The pair \( (U_i, \varphi_i) \) is called a \textbf{chart} and the family $\{(U_i,\varphi_i)\}$ is called an \textbf{atlas}\cite{FredericSchullerTopologicalmanifoldsmanifoldbundlesLec06FredericSchuller2015}.

$M$ is a \textbf{differentiable or smooth manifold} if for any $U_i$ and $U_j$ given that $U_i\cap U_j \neq \emptyset$ the transition function $\varphi_i \circ \varphi_j^{-1}: \varphi_j(U_j \cap U_i) \to \varphi_i(U_j \cap U_i)$ is infinite differentiable ($C^\infty$)\cite{NakaharaGeometrytopologyphysics2005}. In this thesis, smoothness will always be assumed, unless stated otherwise.

Let $M$ and $N$ be two differentiable manifolds of dimension $m$ and $n$ equipped with atlases $\{(U_i, \varphi_i)\}$ and $\{(V_j, \psi_j)\}$ respectively. A map \( f: M \to N \) is called a \textbf{differentiable map} at a point $p \in M$ if for $p \in U_i$ and $f(p) \in V_j$ the composition \( \psi_j \circ f \circ \varphi_i^{-1} : \varphi_i(U_i) \to \psi_j(V_j) \) is infinite differentiable. If $f$ is also a homeomorphism and the inverse \( f^{-1} : N \to M \) is differentiable, then $f$ is called a \textbf{diffeomorphism}. $M$ and $N$ are called \textbf{diffeomorphic} if there exists a diffeomorphism between them. This will be denoted as \( M \cong N \).


\section{Spacetime Manifold $M$}\label{sec:spacetime-manifold}

A trivial but for obvious reasons important example of a differentiable manifold is the spacetime manifold $M$ used in physics, which is defined as follows:

Let $M := \mathbb{R}^4$, the set of ordered 4-tuples $(x^\mu) \in \mathbb{R}^4$. The so called \textbf{standard topology} is defined by the open balls around a point $p \in M$ with radius $r > 0$:
\[
B_r(p) := \{ x \in \mathbb{R}^4 \mid \|x - p\| < r \}
\]
with $\| \cdot \|$ the Euclidean norm:
\[
\|x\|^2 = \sum_{\mu=0}^3 (x^\mu)^2
\]
This is obviously a Hausdorff\footnote{For two distinct points $p,q \in M$ it is always sufficient to choose $r=\frac12\|q - p\|$}, and locally Euclidean topological space. The identity map $\phi(p)=p$ covers $M$ globally. Hence, $(\mathcal{M}, B_r(p), \varphi)$ is a smooth manifold.
