\chapter{Covariant Derivatives}

Covariant derivatives are a fundamental concept in differential geometry and gauge theories, providing a way to differentiate sections of associated vector bundles while respecting the underlying geometric structure. In this chapter, the notion of covariant derivatives will be introduced, and their relationship with connections on principal bundles will be explored.

Recall that sections of bundles associated to a G-principal bundle $\pi : P \rightarrow M$ are equivalent to G-eqivariant functions $\phi : P \rightarrow F$, where F is the typical fiber of the associated bundle. For the definition of the covariant derivative, we first identify the equivarianz condition for finite dimensional linear left actions. This imideatly restricts us to the discussion of associated vector bundles, i.e. $F$ is a finite dimensional vector space.

For any $g\in G$ close to the identity, we can write the equivariant condition as:
$$\phi(p \triangleleft \exp (A t))=\exp (-A t) \triangleright \phi(p)$$
for $A \in T_e G$ and $t \in \mathbb{R}$ small enough. Differentiating both sides with respect to $t$ at $t=0$ yields:
\[
\begin{aligned}
\left.\frac{d}{dt}\,\phi\!\left(p \triangleleft \exp(A t)\right)\right|_{t=0}
&=
  \left.\frac{d}{dt}\,\bigl(\exp(-A t)\,\triangleright\,\phi(p)\bigr)\right|_{t=0} \\[.9pt]
d\phi(p)\!\left(X_p^{A}\right)
&= -\,A \triangleright \phi(p) \\
  &= -\omega(X^A)\triangleright \phi(p)
\end{aligned}
\]
Therefore the G-equivariance condition for linear left actions becomes:
$$d\phi\!\left(X^{A}\right) +  \omega(X^A)\triangleright \phi(p) = 0$$

\section{Definition of Covariant Derivative}

Consider a G-principal bundle $\pi : P \rightarrow M$ with connection form $\omega \in \Omega^1(P, \mathfrak{g})$ and an associated vector bundle $\pi_F : P_F \rightarrow M$ with a section $\sigma \in \Gamma(P_F)$. We now construct a covariant derivative:
$$\nabla_T\sigma \in \Gamma(P_F) \quad \text{for} \, T\in T_xM$$
such that:
\[
\begin{aligned}
  \text{(i)}\;&
\nabla_{\,fT + S}\,\sigma
  = f\,\nabla_T \sigma + \nabla_S \sigma,
  \qquad f \in C^\infty(M),\;\; T,S \in T_xM,
\\[0.8em]
\text{(ii)}\;&
\nabla_T(\sigma + \tau)
  = \nabla_T \sigma + \nabla_T \tau,
\\[0.8em]
\text{(iii)}\;&
\nabla_T(f\sigma)
  = (T f)\,\sigma + f\,\nabla_T \sigma.
\end{aligned}
\]
At this point, most physics literature would derive from these properties the expression for the covariant derivative in a local coordinate patch. However, rather than postulating this expression, we derive it from the underlying geometry. In particular, we show that the covariant derivative emerges naturally from the additional structure that an associated bundle inherits from its principal bundle.

Let $\phi$ be the induced G-equivariant function corresponding to the section $\sigma$. Since it is a function, it is a $F$-valued zero-form on $P$, i.e. $\phi \in \Omega^0(P, F)$. Recall the definition of the exterior covariant derivative $D$ for a $F$-valued k-form $\alpha \in \Omega^k(P, F)$ on $P$, as the exterior derivative $d$ acting only on the horizontal components of vector fields. Thus we define the covariant derivative:
$$D\phi \coloneq d\phi \circ \text{hor}$$
This will take the form:
$$D\phi(X) = d\phi(X) + \omega(X)\wedge\phi \quad \text{for} \, X\in T_pP$$
Since $\phi$ is a zero-form, the wedge product reduces to ordinary multiplication $\omega(X)\wedge\phi = \omega(X)\triangleright\phi$.
To proof this, we consider the cases where $X$ is either horizontal or vertical:
\begin{proof}
(a) Suppose first that $X$ is vertical.  
Then $X = X^{A}$ for some $A \in \mathfrak{g}$, i.e.\ $X$ is the fundamental vector 
field generated by $A$.  
Using the definition of $D\phi$ and the decomposition of $X$ into horizontal 
and vertical components, we have
\[
D\phi(X)
  = d\phi\bigl(\operatorname{hor}(X)\bigr)
  = d\phi(0)
  = 0.
\]
On the other hand, by $G$--equivariance of $\phi$,
\[
d\phi(X^{A}) + \omega(X^A)\,\triangleright\, \phi = 0.
\]
Since $\operatorname{hor}(X)=0$ for vertical $X$.

\medskip

(b) Now let $X$ be horizontal.  
Then $\omega(X)=0$, so we compute
\[
D\phi(X)
  = d\phi\bigl(\operatorname{hor}(X)\bigr)
  = d\phi(X).
\]
\end{proof}
We denote this by $D_X\phi\coloneq D\phi(X)$. This defines a differential operator but on the total space $P$. We obtain the covariant derivative $\nabla_X\sigma$ by introducing a local section $\varphi_i : U_i \subset M \rightarrow P$ of the principal bundle $\pi : P \rightarrow M$ over an open subset $U$ of $M$. Then we can pull back $D_X\phi$ to $U_i$ via $\varphi_i$:
\[
\begin{aligned}
\phi : P \to F 
&\qquad\longrightarrow\qquad 
\phi^* \phi \;=\; \phi \circ \varphi \coloneq s,
\\[0.8em]
\omega \in \Omega^1(P)\otimes T_eG
&\qquad\longrightarrow\qquad
\varphi^* \omega \coloneq \mathcal{A}_i \;\in\; \Omega^1(M)\otimes T_eG,
\\[0.8em]
D\phi \in \Omega^1(P)\otimes F
&\qquad\longrightarrow\qquad
\varphi^*(D\phi)\;\in\; \Omega^1(M)\otimes F.
\end{aligned}
\]
Then we find:
\[
\begin{aligned}
\nabla_Ts \coloneq (\varphi^{*} D\phi)(T)
  &:= \varphi^{*}\bigl( d\phi + \omega \triangleright \phi \bigr)(T) 
\\[0.4em]
  &= \varphi^{*}(d\phi)(T)\;+\;\varphi^{*}(\omega \triangleright \phi)(T)
\\[0.4em]
  &= d(\varphi^{*}\phi)(T)\;+\;\bigl(\varphi^{*}\omega\bigr)(T)\,\triangleright\,(\varphi^{*}\phi)
\\[0.4em]
  &= d s(T)\;+\;\mathcal{A}_i(T)\,\triangleright\, s,
\end{aligned}
\]

