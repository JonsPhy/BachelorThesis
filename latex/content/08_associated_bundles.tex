

\chapter{Associated Bundles}

Associated bundles are a way to construct vector bundles that are associated in a precise manner to a given principal bundle. They play a crucial role in gauge theories, as they provide the mathematical framework for describing matter fields that transform under the action of a gauge group.

\section{Definition of Associated Bundles}

Given a G-principal bundle $\pi : P \rightarrow M$ and a smooth manifold $F$ with a left G-action $\triangleright : G \times F \rightarrow F$, an associated bundle $\pi_F : P_F \rightarrow M$ is defined as follows:

\begin{itemize}
  \item We define the equivalence relation:
  \[
  (p, f) \sim_G (p \triangleleft g, g^{-1} \triangleright f) \quad \forall p \in P, f \in F, g \in G.
  \]
\item The associated bundle $P_F$ is the quotient space:
  $$(P \times F) / \sim_G=: P_F $$
  \item The projection map $\pi_F : P_F \rightarrow M$ is defined by:
  \[
  \pi_F([p, f]) = \pi(p),
  \]
  where $[p, f]$ denotes the equivalence class of $(p, f)$ under the defined relation.
\end{itemize}

It is instructive to recall that the fiber of the principal bundle $P$ is the group $G$ itself. Therefore taking the modulo of $P$ by the group action effectively reproduces the base Manifold. Now roughly speaking, attaching to it a different fiber $F$ with a group action of $G$ results in a new bundle with fiber $F$ over the same base manifold $M$. 

The projection is well-defined, since $\pi_F([p,f]) = \pi(p) = \pi(p \triangleleft g) = \pi_F([p \triangleleft g, g^{-1} \triangleright f ])$ due to the properties of the principal bundle. 

To prove that this construction indeed yields a fiber bundle with typical
fiber $F$, consider a local trivialization of the principal bundle.
Let $\{U_i\}$ be an open cover of $M$ over which $P$ admits smooth
local sections $s_i : U_i \to P$, and let
\[
\phi_i : \pi^{-1}(U_i) \to U_i \times G,
\qquad
p \mapsto \bigl(\pi(p), g_i(p)\bigr),
\]
be the corresponding local trivializations, where
$p = s_i(\pi(p)) \triangleleft g_i(p)$ is unique.
Using this, define a map
\[
\psi_i : U_i \times F \longrightarrow \pi_F^{-1}(U_i), \qquad
\psi_i(x,f) := [\,s_i(x),\, f\,].
\]
This map is smooth and surjective.  
To show that it is a trivialization, define its inverse
\[
\psi_i^{-1}([p,f])
:= \bigl( \pi(p),\, g_i(p)^{-1} \triangleright f \bigr).
\]

This is well-defined: if $(p,f) \sim_G (p \triangleleft g, g^{-1}\triangleright f)$,
then
\[
p = s_i(\pi(p))\triangleleft g_i(p)
\quad\Longrightarrow\quad
p \triangleleft g
= s_i(\pi(p))\triangleleft \bigl(g_i(p)g\bigr),
\]
so the $G$–component transforms as $g_i(p) \mapsto g_i(p)g$, and therefore
\[
(g_i(p)g)^{-1} \triangleright (g^{-1}\triangleright f)
= g^{-1}g_i(p)^{-1} \triangleright g^{-1}\triangleright f
= g_i(p)^{-1}\triangleright f,
\]
so the expression is independent of the chosen representative.
Hence $\psi_i$ is a smooth bijection with smooth inverse, and
\[
\pi_F \circ \psi_i(x,f) = x,
\]
so $(U_i, \psi_i)$ is a local trivialization of $P_F$.
In particular,
\[
\pi_F^{-1}(x) \cong F
\qquad\text{for all } x\in M.
\]

Thus $\pi_F : P_F \to M$ is a smooth fiber bundle with typical fiber $F$.

\subsection{The tangent bundle as an associated bundle}
First, consider the frame bundle $LM$ of a smooth manifold $M$. The frame bundle is a principal $GL(d,\mathbb{R})$-bundle over $M$, where $d = \dim M$. The fiber at each point $x \in M$ consists of all ordered bases (frames) of the tangent space $T_xM$. The right action of $GL(d,\mathbb{R})$ on the frame bundle was defined as:
$$
\begin{aligned}
& \triangleleft: G L(d, \mathbb{R}) \times L M \rightarrow L M \\
& \left(e_1, \ldots, e_d\right) \triangleleft g:=\left(g^m_1 e_m, g^m_2 e_m, \cdots, g^m_d e_m\right)
\end{aligned}
$$

Now construct a associated fiber bundle with typical fiber $\mathbb{R}^d$ and the standard representation of $GL(d,\mathbb{R})$ on $\mathbb{R}^d$. The left action of $GL(d,\mathbb{R})$ on $\mathbb{R}^d$ can be defined as:
$$\triangleright: G L(d, \mathbb{R}) \times F\rightarrow F; \quad (g \triangleright f)^a:= g^a_b f^b$$
Notice that this coincides with the usual definition of how vectors and there components transform under a change of basis. In the discussion of the tangent bundle in chapter 2, this was a condition that was a conclusion. Notice that from the perspective of associated bundles, this is a choice for definition of a left action. As we will see in following sections, this choice is not unique, and different choices lead to different associated bundles.

For the definition given above, the associated bundle is $\pi_{\mathbb{R}^d} : LM_{\mathbb{R}^d} \rightarrow M$. There exists a natural isomorphism $u : LM_{\mathbb{R}^d} \rightarrow TM$
\[
\begin{aligned}
u : LM_{\mathbb{R}^d} \equiv (LM \times \mathbb{R}^d)/{\sim_G}
  &\longrightarrow TM \\[0.6em]
[e,f]
  &\longmapsto f^{a} e_{a}
\end{aligned}
\]
\begin{figure}[htbp]
\centering
\begin{tikzpicture}[scale=1.3, every node/.style={scale=1.1}]
  \matrix (m) [matrix of math nodes, row sep=3em, column sep=4em, ampersand replacement=\&] {
    LM_{\mathbb{R}^d} \& TM \\
    M \& M \\
  };

  \path[->]
    (m-1-1) edge node[left]  {$\pi_{\mathbb{R}^d}$} (m-2-1)
    (m-1-1) edge node[above] {$u$}            (m-1-2)
    (m-1-2) edge node[right] {$\pi_{TM}$}               (m-2-2)
    (m-2-1) edge node[below] {$\mathrm{id}_M$}     (m-2-2);
\end{tikzpicture}
\caption{Commutative diagram exhibiting the tangent bundle $TM$ as the associated bundle
$LM_{\mathbb{R}^d}$ of the frame bundle.}
\end{figure}

This map is invertible, since for any $X \in TM$ any choice of frame $e \in LM$ at $\pi_{TM}(X)$ gives a unique $f \in \mathbb{R}^d$ such that $X = f^a e_a$. The map is well-defined, since for any other representative $(e \triangleleft g, g^{-1} \triangleright f)$ reproduces the equivalence class $[e,f]$.

\subsection{Tensor bundles as associated bundles}
Equivalently to the tangent bundle, one can simply extend this construction to bundles associated to the frame bundle $LM$ with typical fiber
$$
F=\left(\mathbb{R}^d\right)^{p} \times\left(\mathbb{R}^{d^*}\right)^{q}
$$
define a left-actoin $\quad \triangleright: G L(1,1)) \times F \rightarrow F$
by
$$
(g \triangleright f)^{i_1 \cdots i_p}{}_{j_1 \cdots j_q}
  := g^{i_1}{}_{k_1} \cdots g^{i_p}{}_{k_p}
     (g^{-1})^{l_1}{}_{j_1} \cdots (g^{-1})^{l_q}{}_{j_q}\,
     f^{k_1 \cdots k_p}{}_{l_1 \cdots l_q}.
$$
As above this bundle $\pi_F : LM_F \rightarrow M$ is isomorphic to the $(p,q)$-tensor bundle $\pi : T^p_qM \rightarrow M$. This definition formalizes, what in physics is often taken as given. Intuitively, the coordinate transformations is a choice of a different frame in the principal bundle, and each associated bundle must follow the transformation rules defined by the left action of the group on the typical fiber.

\subsection{Tensor densities as associated bundles}
Consider an associated bundle to the frame bundle $LM$ with the same typical fiber as the tensor bundle above
$$
F=\left(\mathbb{R}^d\right)^{p} \times\left(\mathbb{R}^{d^*}\right)^{q}
$$
But now define a different left action $\quad \triangleright: G L(1,1)) \times F \rightarrow F$
by
$$
(g \triangleright f)^{i_1 \cdots i_p}{}_{j_1 \cdots j_q}
  := \left(\det g^{-1}\right)^\omega g^{i_1}{}_{k_1} \cdots g^{i_p}{}_{k_p}
     (g^{-1})^{l_1}{}_{j_1} \cdots (g^{-1})^{l_q}{}_{j_q}\,
     f^{k_1 \cdots k_p}{}_{l_1 \cdots l_q}.
$$
for some $\omega \in \mathbb{Z}$. Then the bundle $\pi_F : LM_F \rightarrow M$ is called the bundle of $(p,q)$-tensor densities of weight $\omega$ over $M$.
Choosing $F=\mathbb{R}$ and $\omega=1$ yields the bundle of scalar densities of 
weight~$1$. The significance of tensor densities becomes clear when considering 
integration on manifolds. Ordinary tensor fields do not transform in a way that 
makes their coordinate expressions compatible with a well-defined notion of 
integration. What \emph{is} integrable on a manifold are \emph{densities}, 
i.e.\ sections of associated bundles whose transition functions include a power 
of the determinant of the Jacobian. The weight~$\omega$ in the $GL(d)$--action 
above precisely controls how such objects respond to a change of frame. For 
instance, scalar densities of weight $1$ transform as
\[
\rho \;\mapsto\; (\det g^{-1})\,\rho ,
\]
ensuring that the combination $\rho\,\mathrm{d}^dx$ is invariantly defined and 
can therefore be integrated over~$M$.

This is directly visible in General Relativity. Under a coordinate transformation with Jacobian matrix
\[
g^{i}{}_{j} = \frac{\partial x^{i}}{\partial x^{\prime j}},
\]
the metric tensor $\gamma$ transforms as
\[
\gamma'_{ij}
  = g^{k}{}_{i}\, g^{l}{}_{j}\, \gamma_{kl},
\]
and therefore its determinant transforms as
\[
\det \gamma'
  = (\det g^{k}{}_{i})^{2}\, \det \gamma
  = (\det g^{-1})^{2} \det \gamma .
\]
Thus $\det \gamma$ is a \emph{scalar density of weight~$2$}, and consequently
$\sqrt{|\det \gamma|}$ is a scalar density of weight~$1$. This is exactly the
object needed to compensate for the transformation of the coordinate measure,
so that
\[
\sqrt{-\gamma}\,\mathrm{d}^{4}x
\]
is invariantly defined. Hence the Einstein--Hilbert action is written as
\[
\int_{M} \sqrt{-\gamma}\, R\, \mathrm{d}^{4}x.
\]
which is independent of any coordinate choice.

The introduction of tensor densities is therefore not merely formal: it 
captures the geometric fact that integration on manifolds is only well 
defined for objects that transform as densities. This also explains why the 
\emph{Lagrangian density} in field theory is called a density: it is a section 
of a weight-$1$ scalar density bundle, ensuring that the action integral is 
well defined. The relation to differential forms and volume elements will be 
discussed further in the context of integration on manifolds.


\section{Associated bundle maps}
Above associated bundles were constructed as fiber bundles which carry additional structure induced by a principal bundle. Therefore we now define maps between associated bundles which preserve this additional structure.
To define these maps consider two associated bundles, with the same fiber $F$, but associated to arbitrary principal bundles $\pi : P \rightarrow M$ and $\pi' : P' \rightarrow M'$ with the same structure group $G$.
An associated bundle map $(\tilde{u},\tilde{h}$ is a bundle map which can be constructed a principal bundle map between the underlying principal bundles $(u,h)$.

\begin{figure}[h!]
\centering
\begin{tikzpicture}[scale=1.3, every node/.style={scale=1.1}]
  \matrix (m) [matrix of math nodes,
               row sep=3.5em,
               column sep=4.5em,
               ampersand replacement=\&] {
    P   \& P'   \\
    P   \& P'   \\
    M   \& M'   \\
  };

  % vertical arrows left column
  \path[->] (m-2-1) edge node[left]  {$\triangleleft G$} (m-1-1);
  \path[->] (m-2-1) edge node[left]  {$\pi$}            (m-3-1);

  % vertical arrows right column
  \path[->] (m-2-2) edge node[right] {$\triangleleft' G$} (m-1-2);
  \path[->] (m-2-2) edge node[right] {$\pi'$}             (m-3-2);

  % horizontal arrows (middle and bottom rows)
  \path[->] (m-1-1) edge node[above] {$u$} (m-1-2);
  \path[->] (m-2-1) edge node[above] {$u$} (m-2-2);
  \path[->] (m-3-1) edge node[above] {$h$}         (m-3-2);
\end{tikzpicture}

\[
\tilde{u}(p \triangleleft g) = \tilde{u}(p)\triangleleft' g,
\qquad
\pi' \circ \tilde{u} = h \circ \pi .
\]

\caption{A morphism of principal $G$-bundles.}
\end{figure}

Then the associated bundle map $(\tilde{u},\tilde{h})$ is defined as:
\[
\begin{aligned}
\tilde{u} : P_F &\longrightarrow P'_F
&\qquad
\tilde{h} : M &\longrightarrow M' \\[0.3em]
  [p,f] &\longmapsto \tilde{u}([p,f])\coloneq [u(p), f]
& m   &\longmapsto h(m)
\end{aligned}
\]
This definition is indeed necessary, since two $F$–fiber bundles may be 
isomorphic as bundles but still fail to be isomorphic as associated 
bundles. This situation occurs when the underlying principal bundles are 
not isomorphic as \emph{principal} $G$–bundles: even though there may exist 
a bundle isomorphism between the total spaces, it might not respect the 
$G$–action. In that case, the formula
\[
[p,f] \longmapsto [u(p),f]
\]
is not well defined on the associated bundles. For 
$(p,f)\sim (p\triangleleft g,\, g^{-1}\triangleright f)$ we obtain
\[
[u(p),f]
\qquad\text{and}\qquad
[u(p\triangleleft g),\, g^{-1}\triangleright f],
\]
and these coincide precisely when
\[
u(p\triangleleft g)=u(p)\triangleleft' g,
\]
i.e.\ when $u$ is $G$–equivariant. Thus $G$–equivariance is exactly the 
condition ensuring that the induced map between associated bundles is 
well defined.

\section{Sections on associated bundles}
This section discusses the relation between sections of associated bundles and equivariant maps on the underlying principal bundle. We will proof the following theorem:

Given a principal $G$-bundle $\pi : P \rightarrow M$ and an associated bundle $\pi_F : P_F \rightarrow M$ with typical fiber $F$. There exists a bijective correspondence between sections $\sigma_F \in \Gamma(P_F)$ of the associated bundle and G-equivariant functions $\phi : P \rightarrow F$ satisfying $\phi(p\triangleleft g) = g^{-1}\triangleright\phi(p)$. 

\begin{proof}
(a) Given a $G$–equivariant map $\phi : P \to F$, define a section
\[
\sigma_\phi : M \longrightarrow P_F,
\qquad
\sigma_\phi(x) \coloneqq [p,\,\phi(p)],
\]
where $p$ is any element of the fiber $\pi^{-1}(x)$.

\medskip

To see that this is well defined, let $p' \in \pi^{-1}(x)$.  
Since $P$ is a principal $G$–bundle, there exists a unique $g \in G$ such that
$p' = p \triangleleft g$. Then
\[
[p',\phi(p')] 
  = [p \triangleleft g,\, \phi(p \triangleleft g)]
  = [p \triangleleft g,\, g^{-1}\!\triangleright \phi(p)]
  = [p,\phi(p)],
\]
where the last equality uses the $G$–equivariance of $\phi$.  
Thus $\sigma_\phi$ is independent of the choice of $p$ and therefore well defined.

\medskip

(b) Conversely, given a section $\sigma : M \to P_F$, we construct a map
$\phi_\sigma : P \to F$.  
For each $p \in P$, the value $\sigma(\pi(p))$ lies in the fiber
$\pi_F^{-1}(\pi(p)) \subset P_F$.  
Define
\[
\phi_\sigma(p) \coloneqq (i_p)^{-1}\bigl(\sigma(\pi(p))\bigr),
\]
where
\[
i_p : F \longrightarrow \pi_F^{-1}(\pi(p)), 
\qquad
i_p(f) \coloneqq [p,f].
\]

The map $i_p$ is a bijection: every element of
$\pi_F^{-1}(\pi(p))$ has the form $[p,f]$ for a unique $f \in F$.  
Furthermore, for any $g \in G$,
\[
i_p(f)
  = [p,f]
  = [p \triangleleft g,\; g^{-1}\!\triangleright f]
  = i_{p\triangleleft g}(g^{-1}\!\triangleright f),
\]
which describes how $i_p$ transforms when $p$ is replaced by another point in
the same principal fiber.

\medskip

We now show that $\phi_\sigma$ is $G$–equivariant. For any $p \in P$ and
$g \in G$,
\[
\begin{aligned}
\phi_\sigma(p \triangleleft g)
  &= (i_{p\triangleleft g})^{-1}\!\bigl(\sigma(\pi(p \triangleleft g))\bigr) \\
  &= (i_{p\triangleleft g})^{-1}\!\bigl(\sigma(\pi(p))\bigr) \\
  &= (i_{p\triangleleft g})^{-1}\!\bigl(i_p(\phi_\sigma(p))\bigr) \\
  &= (i_{p\triangleleft g})^{-1}\!\bigl(i_{p\triangleleft g}(g^{-1}\!\triangleright \phi_\sigma(p))\bigr) \\
  &= g^{-1}\!\triangleright \phi_\sigma(p),
\end{aligned}
\]
using the transformation rule above.  
Thus $\phi_\sigma$ is $G$–equivariant.

\medskip

(c) Finally, we show that the two constructions are inverse to one another.

\smallskip

\noindent\textbf{(1) Recovering $\sigma$ from $\phi_\sigma$.}  
Let $\sigma : M \to P_F$ be a section. For $x \in M$ choose any
$p \in \pi^{-1}(x)$. Then
\[
\begin{aligned}
\sigma_{\phi_\sigma}(x)
  &= [p,\phi_\sigma(p)] \\
  &= \bigl[p,(i_p)^{-1}(\sigma(\pi(p)))\bigr] \\
  &= \sigma(\pi(p)) \\
  &= \sigma(x),
\end{aligned}
\]
because $\pi(p)=x$ and $i_p$ is a bijection onto the fiber over $x$.

\smallskip

\noindent\textbf{(2) Recovering $\phi$ from $\sigma_\phi$.}  
Let $\phi : P \to F$ be $G$–equivariant. Then for any $p \in P$,
\[
\begin{aligned}
\phi_{\sigma_\phi}(p)
  &= (i_p)^{-1}\!\bigl(\sigma_\phi(\pi(p))\bigr) \\
  &= (i_p)^{-1}\!\bigl([p,\phi(p)]\bigr).
\end{aligned}
\]
Since $i_p(f) = [p,f]$, the only $f$ satisfying $[p,f] = [p,\phi(p)]$ is
$f=\phi(p)$. Therefore
\[
\phi_{\sigma_\phi}(p) = \phi(p),
\]
so $\phi_{\sigma_\phi} = \phi$.

\medskip

We conclude that
\[
\phi \longmapsto \sigma_\phi,
\qquad
\sigma \longmapsto \phi_\sigma
\]
are mutually inverse bijections between $G$–equivariant maps $P \to F$ and
sections of the associated bundle $P_F$.
\end{proof}
