
\chapter{Gauge Theories}

In physical gauge theories like electromagnetism, Yang-Mills theories or general relativity, the laws of nature they describe are not just differential equations that happen to describe nature, but they are deeply connected to the geometry of the underlying symmetries. In the following, the above developed mathematical framework is applied to recover Maxwell's equations, Yang-Mills theories.

\section{Maxwell theory}

Consider a $U(1)$ principal bundle $P$ over the four dimensional Minkowski spacteime manifold $M$ equipped with the Minkowski metric $\eta = \eta_{\mu\nu}dx^\mu \otimes dx^\nu$. The principal bundle is trivial $P = M \times U(1)$, and the projection map is given by $\pi: P \to M$, $\pi(x,e^{i\Lambda}) = x$. The Yang-Mills field is given by:

\[ \mathcal{A}= \mathcal{A}_\mu dx^\mu \in \Omega^1(M)\otimes \mathfrak{u}(1) \]
And the field strength is given by the curvature two-form:
\[ \mathscr{F} = d \mathcal{A} \]

We identify the \textbf{gauge potential} $A$ by $\mathcal{A}_\mu=iA_\mu$ and the \text{field strength tensor} $F$ by $\mathscr{F}_{\mu\nu} = iF_{\mu\nu}$, where $i$ is the factor associated with the Lie algebra. Therefore, the curvature two-form can be written in components as:
\[ F_{\mu\nu} = \partial_\mu A_\nu - \partial_\nu A_\mu \]

The Bianchi identity is given by:

\begin{align*}
  D\mathscr{F} &= d\mathscr{F} \\
  &= \frac12\partial_\mu \mathscr{F}_{\nu\rho} \, dx^\mu \wedge dx^\nu \wedge dx^\rho = 0 \\
  \Rightarrow & \quad \partial_\mu \, \mathscr{F}_{\nu\rho} + \partial_\nu \, \mathscr{F}_{\rho\mu} + \partial_\rho \mathscr{F}_{\mu\nu} = 0
\end{align*}


When identifying the electric and magnetic fields with the components of the field strength tensor, we have:
\begin{align*}
  E_i &= F_{0i}\\
  B_i &= \frac12 \epsilon_{ijk} F_{jk} 
\end{align*}

The Bianchi identity yields the \textbf{homogeneous Maxwell equations}:
\begin{align*}
  \partial_\mu F_{\nu\rho} + \partial_\nu F_{\rho\mu} + \partial_\rho F_{\mu\nu} &= 0
\end{align*}

First, choosing indices \(\mu=0\), \(\nu=i\), \(\rho=j\) and using antisymmetry \(F_{\mu\nu} = -F_{\nu\mu}\), we obtain:
\begin{align*}
  \partial_0 F_{ij} + \partial_i F_{j0} + \partial_j F_{0i} &= 0 \\
  \Rightarrow \quad (\nabla \times \mathbf{E})_k + \partial_t B_k &= 0
\end{align*}

Now, choose \(\mu=i\), \(\nu=j\), \(\rho=k\), all spatial indices. Contracting with the Levi-Civita tensor \(\epsilon^{ijk}\) gives:
\begin{align*}
  \epsilon^{ijk} \partial_i F_{jk} &= 0 \\
  \Rightarrow \quad \nabla \cdot \mathbf{B} &= 0
\end{align*}

Together, these two identities form the homogeneous Maxwell equations:
\begin{align*}
  \nabla \times \mathbf{E} + \partial_t \mathbf{B} &= 0 \\
  \nabla \cdot \mathbf{B} &= 0
\end{align*}



\section{Yang-Mills theory}

The same construction applies to non-Abelian gauge theories with structure group \( G = SU(N) \). We consider a trivial principal bundle over the spacteime manifold \( P = M \times SU(N) \), with projection map \( \pi: P \to M \), \( \pi(x, g) = x \). The Yang–Mills field is given by a \( \mathfrak{su}(N) \)-valued one-form:
\[
\mathcal{A} = A_\mu^a T^a dx^\mu,
\]
where \( T^a \in \mathfrak{su}(N) \) are the generators of the Lie algebra of \( SU(N) \), and \( A_\mu^a \) are the gauge potentials. The corresponding field strength (curvature two-form) is given by:
\[
\mathscr{F} = d\mathcal{A} + \mathcal{A} \wedge \mathcal{A}.
\]
In components, this reads:
\begin{align*}
  \mathscr{F}_{\mu\nu} 
  &= \partial_\mu A_\nu^a T^a - \partial_\nu A_\mu^a T^a + A_\mu^b A_\nu^c [T^b, T^c] \\[0.8em]
  &= \left( \partial_\mu A_\nu^a - \partial_\nu A_\mu^a + f^{abc} A_\mu^b A_\nu^c \right) T^a 
  =: F_{\mu\nu}^a T^a,
\end{align*}
where \( f^{abc} \) are the structure constants of \( \mathfrak{su}(N) \), defined via \( [T^b, T^c] = f^{abc} T^a \).

The Bianchi identity holds:
\[
D\mathscr{F} = d\mathscr{F} + [\mathcal{A}, \mathscr{F}] = 0,
\]
which in components becomes:
\[
D_\lambda F_{\mu\nu}^a 
= \partial_\lambda F_{\mu\nu}^a + f^{abc} A_\lambda^b F_{\mu\nu}^c = 0.
\]


\section{Gauge sector of the standard model of particle physics}

The above elaboration of gauge theories can be applied to the gauge sector of the standard model of particle physics. It is described by a principal bundle with the gauge group \( SU(3) \times SU(2) \times U(1) \). Combining the above results, the action describing the dynamics of the gauge fields may be written as:
\[
S_{\text{gauge}} =  \int_M d^4x -\frac{1}{4}\left( G_{\mu\nu}^{a} G^{a\,\mu\nu} + W_{\mu\nu}^{i} W^{i\,\mu\nu} + B_{\mu\nu} B^{\mu\nu} \right) \, \mathrm{d}^4x
\]
where \( G_{\mu\nu}^a \), \( W_{\mu\nu}^i \), and \( B_{\mu\nu} \) denote the field strength components of the curvature two-forms corresponding to \( SU(3) \), \( SU(2) \), and \( U(1) \), respectively. These arise from the decomposition of the total curvature \( \mathscr{F} \in \mathfrak{g} \otimes \Omega^2(M) \). 

