
\section{Gauge Theories}

In physical gauge theories like electromagnetism, Yang-Mills theories or general relativity, the laws of nature they describe are not just differential equations that happen to describe nature, but they are deeply connected to the geometry of the underlying symmetries. In the following, the above developed mathematical framework is applied to recover Maxwell's equations, Yang-Mills theories.

\subsection{Maxwell theory}

Consider a $U(1)$ principal bundle $P$ over the four dimensional Minkowski spacteime manifold $M$ equipped with the Minkowski metric $\eta = \eta_{\mu\nu}dx^\mu \otimes dx^\nu$. The principal bundle is trivial $P = M \times U(1)$, and the projection map is given by $\pi: P \to M$, $\pi(x,e^{i\Lambda}) = x$. The Yang-Mills field is given by:

\[ \mathcal{A}= \mathcal{A}_\mu dx^\mu \in \Omega^1(M)\otimes \mathfrak{u}(1) \]
And the field strength is given by the curvature two-form:
\[ \mathscr{F} = d \mathcal{A} \]

We identify the \textbf{gauge potential} $A$ by $\mathcal{A}_\mu=iA_\mu$

\subsection{Yang-Mills theory}
