
\chapter{Principal Bundles}

A principal bundle is a fiber bundle $P \xrightarrow{\pi} M$ is a bundle whose fiber is identical to the structure group. This framework is of particular importance, because it allows to understand fibre bundles with fibre $F$ on which $G$ acts. These bundles are called \emph{associated bundles} and are essential to understand Gauge theories in physics.

\section{Action of a Lie Group on a Manifold}\label{sec:action-of-lie-group-on-manifold}

To understand this properly we first need to understand how a Lie group $G$ can act on a manifold $M$ \cite{FredericSchullerPrincipalfibrebundlesLec19FredericSchuller2015}
Let \( (G, \cdot) \) be a Lie group, and \( M \) a smooth manifold.  
Then a smooth map
\[
\triangleright : G \times M \longrightarrow M
\]
satisfying
\begin{align*}
  e \triangleright p &= p && \forall p \in M, \text{ and $e$ being the identity in $G$} \\
  g_2 \triangleright (g_1 \triangleright p) &= (g_2 \cdot g_1) \triangleright p && \forall g_1, g_2 \in G, \, p \in M
\end{align*}
is called a \textbf{left \( G \)-action} on \( M \).

Analogous a \textbf{right \( G \)-action} $\triangleleft : M \times G \longrightarrow M$ is defined, satisfying:
\begin{align*}
p \triangleleft e &= p && \forall p \in M, \text{ and $e$ being the identity in $G$} \\
  (p \triangleleft g_1) \triangleleft g_2 &= p \triangleleft (g_1 g_2) && \forall g_1, g_2 \in G, \, p \in M
\end{align*}

Given a left action $\triangleright$, we can construct a right action: 
\begin{align*}
  \triangleleft : &M \times G \longrightarrow M, \\
  & p \triangleleft g := g^{-1} \triangleright p.
\end{align*}

It is trivial to show that this yields a right action.

Let \( G \) be a Lie group acting smoothly on a manifold \( M \) from the left via
\[
\triangleright : G \times M \longrightarrow M.
\]

We define an equivalence relation \( \sim \) on \( M \) by:

\begin{align*}
p \sim \tilde{p} \quad :\Longleftrightarrow \quad \exists g \in G \text{ such that } \tilde{p} = g \triangleright p.
\end{align*}

This defines an equivalence relation:
\begin{itemize}
  \item \textbf{Reflexivity:} \( e \triangleright p = p \), so \( p \sim p \).
  \item \textbf{Symmetry:} If \( \tilde{p} = g \triangleright p \), then \( p = g^{-1} \triangleright \tilde{p} \), so \( \tilde{p} \sim p \).
  \item \textbf{Transitivity:} If \( \tilde{p} = g_1 \triangleright p \) and \( \hat{p} = g_2 \triangleright \tilde{p} \), then
  \[
  \hat{p} = g_2 \triangleright (g_1 \triangleright p) = (g_2 g_1) \triangleright p,
  \]
  so \( p \sim \hat{p} \).
\end{itemize}

The \textbf{orbit} of a point \( p \in M \) under the group action is then the equivalence class:
\[
\mathcal{O}_p := \{ \tilde{p} \in M \mid \exists g \in G : \tilde{p} = g \triangleright p \}.
\]

We can then define the \textbf{quotient space} \( M/\!\sim \), often denoted \( M/G \), by identifyinf points that are in the same orbit.

We define the \textbf{stabilizer} of a point \( p \in M \) as the set of elements in \( G \) that leave \( p \) unchanged:
\[ S_p \coloneqq \{g \in G \mid g \triangleright p = p \} \]

An action $\triangleright$ is called free if for all \( p \in M \) the stabilizer is trivial \( S_p = \{e\} \).





\section{Principal Bundles}

We now define a principal fibre bundle as follows \cite{FredericSchullerPrincipalfibrebundlesLec19FredericSchuller2015} 

Let \( (P, \pi, M, F) \) be a fibre bundle. If the following conditions are satisfied, we call it a \textbf{principal \( G \)-bundle}:

\begin{align*}
  \text{(i)}\quad & P \text{ is equipped with a right } G\text{-action } \triangleleft, \\
  \text{(ii)}\quad & \text{The action of } G \text{ is free}, \\
  \text{(iii)}\quad & \pi : P \to M \text{ is isomorphic as a bundle to the quotient } \rho : P \to P/G,
\end{align*}

where \( \rho(p) \mapsto [p] \) denotes the canonical projection onto the orbit space \( P/G \). 

To clarify, two bundles \( \pi: E \to M \) and \( \pi': E' \to M' \) are isomorphic if there exist diffeomorphisms \( \bar{f}: E \to E' \) and \( f: M \to M' \) such that \( \pi' \circ \bar{f} = f \circ \pi \).

Since \( \triangleleft \) acts freely, each fibre \( \rho^{-1}([p]) \) is diffeomorphic to \(G\).


\paragraph{The Frame Bundle}

As an example we consider the frame bundle of a smooth manifold \( M \).

We define a \textbf{frame} at a point \(p\in M\) with \(\text{dim} M = d\) as an ordered basis of the tangent space \(T_pM\) and the set  of all frames at \(p\) as:
\[
L_pM \coloneqq \left\{ (e_1, \dots, e_d) \,\middle|\, \{e_1, \dots, e_d\} \text{ is a basis of } T_pM \right\}
\]

There exists a natural isomorphism
\[
L_pM \cong \mathrm{GL}(d, \mathbb{R}),
\]
by identifying each frame \( (e_1, \dots, e_d) \in L_pM \) with the matrix whose columns are the components of the vectors \( e_i \). Concretely, each frame is mapped to a matrix \( g \in \mathrm{GL}(d, \mathbb{R}) \) such that
\[
g^\mu_{\;\;\alpha} = e^\mu_{\;\;\alpha}
\]



Analogous to the tangent bundle, we can define the \textbf{frame bundle}:
\[ LM := \bigsqcup_{p \in M} L_pM \]

Given a chart \(U_i\) on \( M \), we can define a local trivialization of the frame bundle \cite{NakaharaGeometrytopologyphysics2005}. A frame $\epsilon = \{e_1, \dots, e_d\}$ at $p\in M$ is expressed in terms of the natural basis of the tangent space \(T_pM\) \( \left\{ \partial / \partial x^\mu \mid_p \right\} \)
\[ e_\alpha = e^\mu_{\,\,\alpha} \, \partial/\partial x^\mu \mid_p \quad \text{where }e^\mu_{\,\,\alpha} \in \mathrm{GL}(d, \mathbb{R}) \]
The local trivialization is then given by \( \varphi_i^{-1}(u)=(p,(e^\mu_{\,\,\alpha}))\).

The projection $\pi_L$ of a frame $\epsilon = \{e_1, \cdot, e_d\}$ at a point \(p\in M\) is given by:
\begin{align*}
  \pi_L \, : \, LM &\longrightarrow M, \\
        \epsilon &\mapsto \pi_L(\epsilon)=p
\end{align*}

Therefore we have introduced the necessary structure, suck that $(LM, \pi, M, \text{GL}(d,\mathrm{R})$ defines a fiber bundle. 

The right action of \( \mathrm{GL}(d,\mathbb{R}) \) on the frame is defined as we would expect, since it is analogguss to the change of basis in a vector space:
\begin{align*}
  \triangleleft : LM \times \mathrm{GL}(d,\mathbb{R}) &\longrightarrow LM, \\
  (\epsilon, g) &\mapsto \epsilon \triangleleft g = (e_1, \cdots, e_d) \triangleleft g = (e_i g^i_{\,\,1}, \cdots, e_i g^i_{\,\,d}),
\end{align*}

To show that this Bundle equipped with this right action is a principal bundle, we need to check, if the action is free and the bundle is isomorphic to the quotient bundle.

Therefore we need to show that \( e_i g^i_{\,\,\alpha} = e_\alpha \implies g^i_{\,\,\alpha} = \delta^i_{\,\,\alpha} \, \forall \alpha \) . Since \( \{e_1, \dots, e_d\} \) is a basis, they are linearly independent. Therefore the definition already implies that \( g^i_{\,\,\alpha} = \delta^i_{\,\,\alpha} \) and thus the action is free.

To show the second condition, we need to show that the orbit space \( LM/\mathrm{GL}(d,\mathbb{R}) \) is consists of a single point for each \( \epsilon \in \pi^{-1}(p) \). This is indeed the case, since the orbit of a frame at \( p \in M \) under the action of \( \mathrm{GL}(d,\mathbb{R}) \) consists of all frames at \( p \), since the action is transitiv. Thus the quotient space is diffeomorphic to \( M \), with the diffeomorphism being: \( f: M \longrightarrow LM/\mathrm{GL}(d, \mathbb{R}), \, p \mapsto [\epsilon] \) where \( \epsilon \) is a frame at \( p \).


