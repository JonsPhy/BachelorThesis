
\chapter{Principal Bundles}

A principal bundle is a fiber bundle \(P \xrightarrow{\pi} M\) characterized by a fiber that is identical to the structure group. This framework is particularly significant as it facilitates the underlying structure of gauge theories in physics, where the fiber represents the gauge group and the base manifold \(M\) represents spacetime. In this chapter, it will first be elaborated how a Lie group acts on a manifold. Then, the definition of a principal bundle will be given, followed by an example of the frame bundle.

\section{Action of a Lie Group on a Manifold}\label{sec:action-of-lie-group-on-manifold}

To understand this concept thoroughly, it is necessary to examine how a Lie group \(G\) can act on a manifold \(M\) \cite{FredericSchullerPrincipalfibrebundlesLec19FredericSchuller2015}\cite{DudekEhreshmanntheoryconnectionprincipalbundlecompendiumphysicists2018}. Let \( (G, \cdot) \) denote a Lie group, and \(M\) a smooth manifold. A smooth map

\[
\triangleright : G \times M \longrightarrow M
\]

is defined as a \textbf{left \(G\)-action} on \(M\) if it satisfies the following conditions:

\begin{align*}
  e \triangleright p &= p && \forall p \in M, \text{ where $e$ is the identity in $G$} \\
  g_2 \triangleright (g_1 \triangleright p) &= (g_2 \cdot g_1) \triangleright p && \forall g_1, g_2 \in G, \, p \in M.
\end{align*}

Analogously, a \textbf{right \(G\)-action} \(\triangleleft\) is defined by the map \(\triangleleft : M \times G \longrightarrow M\), which satisfies:

\begin{align*}
p \triangleleft e &= p && \forall p \in M, \text{ where $e$ is the identity in $G$} \\
(p \triangleleft g_1) \triangleleft g_2 &= p \triangleleft (g_1 \cdot g_2) && \forall g_1, g_2 \in G, \, p \in M.
\end{align*}

Given a left action \(\triangleright\), it is possible to construct a right action as follows:

\begin{align*}
\triangleleft : &M \times G \longrightarrow M, \\
& p \triangleleft g := g^{-1} \triangleright p.
\end{align*}

This transformation yields a valid right action since the inverse of the identity is again the identity and $(g_1 \cdot g_2)^{-1} = g_2^{-1} \cdot g_1^{-1}$.

Let \(G\) be a Lie group acting smoothly on a manifold \(M\) from the left via

\[
\triangleright : G \times M \longrightarrow M.
\]

An equivalence relation \(\sim\) on \(M\) is defined by:

\begin{align*}
p \sim \tilde{p} \quad :\Longleftrightarrow \quad \exists g \in G \text{ such that } \tilde{p} = g \triangleright p.
\end{align*}

This equivalence relation can be verified through the following properties:

\begin{itemize}
  \item \textbf{Reflexivity:} \(e \triangleright p = p\), thus \(p \sim p\).
  \item \textbf{Symmetry:} If \(\tilde{p} = g \triangleright p\), then \(p = g^{-1} \triangleright \tilde{p}\), leading to \(\tilde{p} \sim p\).
  \item \textbf{Transitivity:} If \(\tilde{p} = g_1 \triangleright p\) and \(\hat{p} = g_2 \triangleright \tilde{p}\), then

\[
\hat{p} = g_2 \triangleright (g_1 \triangleright p) = (g_2 g_1) \triangleright p,
\]
which implies \(p \sim \hat{p}\).
\end{itemize}

The \textbf{orbit} of a point \(p \in M\) under the group action constitutes the equivalence class:

\[
\mathcal{O}_p := \{ \tilde{p} \in M \mid \exists g \in G : \tilde{p} = g \triangleright p \}.
\]

Consequently, the \textbf{quotient space} \(M/\!\sim\), commonly denoted as \(M/G\), is defined by identifying points within the same orbit.

Additionally, the \textbf{stabilizer} of a point \(p \in M\) represents the set of elements in \(G\) that leave \(p\) unchanged:

\[
S_p \coloneqq \{g \in G \mid g \triangleright p = p \}.
\]

An action \(\triangleright\) is termed free if the stabilizer is trivial for all \(p \in M\), i.e., \(S_p = \{e\}\).


\section{Principal Bundles}

A principal fibre bundle is defined as follows \cite{DudekEhreshmanntheoryconnectionprincipalbundlecompendiumphysicists2018}:

Let \( (P, \pi, M, F) \) be a fibre bundle. If the following conditions are satisfied, it is classified as a \textbf{principal \( G \)-bundle}:

\begin{align*}
  \text{(i)}\quad & P \text{ is equipped with a right } G\text{-action } \triangleleft, \\
  \text{(ii)}\quad & \text{The action of } G \text{ is free}, \\
  \text{(iii)}\quad & \pi : P \to M \text{ is isomorphic as a bundle to the quotient } \rho : P \to P/G,
\end{align*}

where \( \rho(p) \mapsto [p] \) denotes the canonical projection onto the orbit space \( P/G \). Since the action \( \triangleleft \) is free, each fibre \( \rho^{-1}([p]) \) is diffeomorphic to \( G \).

For clarification, two bundles \( \pi: E \to M \) and \( \pi': E' \to M' \) are considered isomorphic if there exist diffeomorphisms \( \bar{f}: E \to E' \) and \( f: M \to M' \) such that \( \pi' \circ \bar{f} = f \circ \pi \). Thus, (iii) can be reformulated as the existence of a diffeomorphism \( f : M \to P/G \) such that the following diagram commutes:
\begin{figure}[h!]
\centering
\begin{tikzpicture}[scale=1.3, every node/.style={scale=1.1}]
  \matrix (m) [matrix of math nodes, row sep=3em, column sep=4em, ampersand replacement=\&] {
    P \& P \\
    M \& P/G \\
  };

  \path[->] 
    (m-1-1) edge node[left] {\(\pi\)} (m-2-1)
    (m-1-1) edge node[above] {\(\mathrm{id}_P\)} (m-1-2)
    (m-1-2) edge node[right] {\(\rho\)} (m-2-2)
    (m-2-1) edge node[above] {\(f\)} (m-2-2);
\end{tikzpicture}
\caption{Commutative diagram showing the isomorphism between a principal $G$-bundle \(\pi : P \to M\) and its quotient bundle \(\rho : P \to P/G\).}
\end{figure}


\section{The Frame Bundle}

As an example, consider the frame bundle of a smooth manifold \( M \)\cite{NakaharaGeometrytopologyphysics2005}.

A \textbf{frame} at a point \( p \in M \) with \(\text{dim} M = d\) is defined as an ordered basis of the tangent space \(T_pM\). The set of all frames at \( p \) is given by:
\[
L_pM \coloneqq \left\{ (e_1, \dots, e_d) \,\middle|\, \{e_1, \dots, e_d\} \text{ is a basis of } T_pM \right\}.
\]

There exists a natural isomorphism
\[
L_pM \cong \mathrm{GL}(d, \mathbb{R}),
\]
by identifying each frame \( (e_1, \dots, e_d) \in L_pM \) with the matrix whose columns are the vectors \( e_i \). Specifically, each frame is mapped to a matrix \( g \in \mathrm{GL}(d, \mathbb{R}) \) such that
\[
g^\mu_{\;\;\alpha} = e^\mu_{\;\;\alpha}.
\]

Analogous to the tangent bundle, the \textbf{frame bundle} can be defined as:
\[ LM := \bigsqcup_{p \in M} L_pM. \]

Given a chart \( U_i \) on \( M \), a local trivialization of the frame bundle can be defined \cite{NakaharaGeometrytopologyphysics2005}. A frame \( \epsilon = (e_1, \dots, e_d) \) at \( p \in M \) can be expressed in terms of the natural basis of the tangent space \( T_pM \) as \( \left\{ \partial / \partial x^\mu \mid_p \right\} \):
\[ e_\alpha = e^\mu_{\,\,\alpha} \, \partial/\partial x^\mu \mid_p, \quad \text{where } e^\mu_{\,\,\alpha} \in \mathrm{GL}(d, \mathbb{R}). \]
The local trivialization is then given by \( \varphi_i^{-1}(u)=(p,(e^\mu_{\,\,\alpha})) \).

The projection \( \pi_L \) of a frame \( \epsilon = \{e_1, \cdots, e_d\} \) at a point \( p \in M \) is specified by:
\begin{align*}
  \pi_L \, : \, LM & \longrightarrow M, \\
        \epsilon & \mapsto \pi_L(\epsilon)=p.
\end{align*}

Thus, the necessary structure has been introduced such that \( (LM, \pi_L, M, \text{GL}(d,\mathbb{R})) \) defines a fiber bundle.

The right action of \( \mathrm{GL}(d,\mathbb{R}) \) on the frame is analogous to a change of basis in a vector space. This action is defined as follows:
\begin{align*}
  \triangleleft : LM \times \mathrm{GL}(d,\mathbb{R}) & \longrightarrow LM, \\
  (\epsilon, g) & \mapsto \epsilon \triangleleft g = (e_1, \cdots, e_d) \triangleleft g = (e_i g^i_{\,\,1}, \cdots, e_i g^i_{\,\,d}).
\end{align*}

To demonstrate that this bundle, equipped with the right action, is a principal bundle, it is necessary to verify that the action is free and that the bundle is isomorphic to the quotient bundle.

First, it must be shown that \( e_i g^i_{\,\,\alpha} = e_\alpha \implies g^i_{\,\,\alpha} = \delta^i_{\,\,\alpha} \, \forall \alpha \). Since \( \{e_1, \dots, e_d\} \) forms a basis, the vectors are linearly independent. Therefore, the definition implies that \( g^i_{\,\,\alpha} = \delta^i_{\,\,\alpha} \), confirming that the action is free.

Next, it is necessary to show that the orbit space \( LM/\mathrm{GL}(d,\mathbb{R}) \) consists of a single point for each \( \epsilon \in \pi^{-1}(p) \). This is indeed the case, as the orbit of a frame at \( p \in M \) under the action of \( \mathrm{GL}(d,\mathbb{R}) \) includes all frames at \( p \), since the action is transitive. Hence, the quotient space is diffeomorphic to \( M \), with the diffeomorphism defined by: \( f: M \longrightarrow LM/\mathrm{GL}(d,\mathbb{R}), \, p \mapsto [\epsilon] \), where \( \epsilon \) is a frame at \( p \).

Thus, the frame bundle \( (LM, \pi_L, M, \mathrm{GL}(d,\mathbb{R})) \) is indeed a principal \( \mathrm{GL}(d,\mathbb{R}) \)-bundle.
