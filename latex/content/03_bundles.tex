
\section{Bundles}


\subsection{Tangent Space $\mathcal{T}_p\mathcal{M}$}

  The definition of a vector on a Manifold is non trivial, because a vector space structure might not exist globally on the manifold.

  Let $\mathcal{M}$ be an $n$-dimensional smooth manifold. A tangent vector at a point $p \in \mathcal{M}$ is a linear map $v: C^\infty(\mathcal{M}) \to \mathbb{R}$ satisfying the Leibniz rule\footnote{$f \in C^\infty(\mathcal{M})$ if for all charts $(U, \varphi)$ on $\mathcal{M}$, the composition $f \circ \varphi^{-1}: \varphi(U) \subset \mathbb{R}^n \to \mathbb{R}$ is infinitely differentiable.}:
  \[ v[fg] = v[f]g(p) + f(p)v[g] \]
  
  Such a map is called a \textit{derivation} at $p$. The set of all such derivations forms a vector space, the \textit{tangent space} at $p$, denoted $\mathcal{T}_p\mathcal{M}$.

  We produce a tangent vector at a point $p \in \mathcal{M}$ as an equivalence class of curves through $p$.

  Let $\gamma_n: [-\epsilon, \epsilon] \to \mathcal{M}$ be a family of smooth curves in $\mathcal{M}$. Then $\varphi(\gamma_n(t)) = x_n^\mu(t) \in \mathbb{R}^n$ is called the coordinate representation of $\gamma_n$ in a chart $(U, \varphi)$.

Let $f \in C^\infty(\mathbb{R}^n)$, then we can define
\begin{align*}
\left. \frac{d}{dt} (f \circ \gamma)(t) \right|_{t=0}
&= \left. \frac{d}{dt} \left( f(x^\mu(t)) \right) \right|_{t=0} \\
&= \left. \frac{\partial f}{\partial x^r} \frac{d x^r}{dt} \right|_{t=0} \\
&= \left. \frac{d x^r}{dt} \right|_{t=0} \left. \frac{\partial f}{\partial x^r} \right|_p
\end{align*}

We define an equivalence class.  
We say two curves are equivalent  
if
\[
\left. \frac{d}{dt} \left( f \circ \gamma_1 \right)(t) \right|_{t=0}
= \left. \frac{d}{dt} \left( f \circ \gamma_2 \right)(t) \right|_{t=0}
\]

For two equivalent curves $\gamma_1$ and $\gamma_2$,
\[
\left. \frac{d x_1^r}{dt} \right|_{t=0}
= \left. \frac{d x_2^r}{dt} \right|_{t=0}
= v^r
\]

A tangent vector is
\[
v = v^r \left. \frac{\partial}{\partial x^r} \right|_p
\]
We notice: Once a chart $(U, \varphi)$ is chosen, with local coordinates $(x^1, \dots, x^n)$, a tangent vector is represented as a linear combination of partial derivatives with real coefficients.

\[
\left\{ \left. \frac{\partial}{\partial x^1} \right|_p, \dots, \left. \frac{\partial}{\partial x^n} \right|_p \right\} \quad \text{form a basis of } T_p \mathcal{M}
\]



\subsection*{The Tangent Bundle as a Fiber Bundle}

To continue, we need to introduce fiber bundles. To do so, we first take a look at a specific example: the tangent bundle.

To construct the tangent bundle of a smooth $n$-dimensional manifold $\mathcal{M}$, we take the disjoint union of all tangent spaces $T_p\mathcal{M}$:
\[
T\mathcal{M} := \bigsqcup_{p \in \mathcal{M}} T_p\mathcal{M}
= \bigcup_{p \in \mathcal{M}} \{p\} \times T_p\mathcal{M}
\]

Each element of $T\mathcal{M}$ is a pair $(p, v)$, where $p \in \mathcal{M}$ is a point in the base space and $v \in T_p\mathcal{M}$ is a tangent vector at that point.

We define the natural projection:
\[
\pi: T\mathcal{M} \to \mathcal{M}, \quad (p, v) \mapsto p
\]
which "forgets" the tangent vector.

We call $\pi^{-1}(p) = T_p\mathcal{M}$ the fiber over $p$. Since $T_p\mathcal{M} \cong \mathbb{R}^n$ as a vector space, we refer to the model fiber as $F = \mathbb{R}^n$.

The final step is to give $T\mathcal{M}$ a smooth structure of dimension $2n$, we proceed as follows:  
Given a coordinate chart $(U, \varphi)$ on $\mathcal{M}$, we define a chart on $\pi^{-1}(U) \subset T\mathcal{M}$ by:
\[
\Psi: \pi^{-1}(U) \to \mathbb{R}^{2n}, \quad (p, v) \mapsto \left(x^1(p), \dots, x^n(p), v^1, \dots, v^n \right)
\]
where $v = v^i \left. \frac{\partial}{\partial x^i} \right|_p$.

This defines a local trivialization:
\[
T\mathcal{M}|_U \cong U \times \mathbb{R}^n
\]

Thus, we have constructed a fiber bundle with:
\begin{itemize}
  \item \textbf{Total space:} $T\mathcal{M} = \bigsqcup_{p \in \mathcal{M}} T_p\mathcal{M}$
  \item \textbf{Base space:} $\mathcal{M}$
  \item \textbf{Projection:} $\pi: T\mathcal{M} \to \mathcal{M}$
  \item \textbf{Model fiber:} $F = \mathbb{R}^n$
  \item \textbf{Local trivialization:} $T\mathcal{M}|_U \cong U \times \mathbb{R}^n$
\end{itemize}

This provides an intuitive picture of fiber bundles:  
Analogous to a manifold, which we often think of as a space that locally resembles $\mathbb{R}^n$, a fiber bundle can be thought of as a space that locally resembles a product of the base space with a typical fiber\footnote{Often the product of two manifolds, but not necessarily so}.

\subsection*{General Definition of a Fiber Bundle}

The formal definition of a fiber bundle reads as follows.
A fiber bundle is a quadruple $(E, B, \pi, F)$ where:
\begin{itemize}
  \item $E$ is the total space
  \item $B$ is the base space
  \item $\pi: E \to B$ is a surjective map called the projection
  \item $F$ is the typical fiber
\end{itemize}

There exists an open cover $\{U_\alpha\}$ of $B$ such that for each $\alpha$, there is a diffeomorphism
\[
\varphi_\alpha: \pi^{-1}(U_\alpha) \to U_\alpha \times F
\]
such that the following diagram commutes:


\begin{center}
\begin{tikzpicture}[scale=1.5, every node/.style={font=\normalsize}]
  \node (A) at (0,1.5) {$\pi^{-1}(U)$};
  \node (B) at (4,1.5) {$U \times F$};
  \node (C) at (0,0) {$U$};

  \draw[->] (A) -- (B) node[midway, above] {$\varphi$};
  \draw[->] (A) -- (C) node[midway, left] {$\pi$};
  \draw[->] (B) -- (C) node[midway, right] {$\mathrm{proj}_1$};
\end{tikzpicture}
\end{center}


where $\text{proj}_1: A \times B \to A$ is the first projection.

$(U_\alpha, \varphi_\alpha)$ are called \emph{local trivialization}.

The fiber over a point $b \in B$ is:
\[
F_b := \pi^{-1}(\{b\}) \cong F
\]

We often denote a fiber bundle as $E \xrightarrow{\pi} B$

\subsection*{Sections}

One important definition in the context of fiber bundles is that of a section or cross-section. This allows us to pick an element from each fiber over each point in a continuous manner, intoducing concepts like vector fields over space time.
A \emph{section} of a fiber bundle is a continuous map $s: B \to E$ such that
\[
\pi \circ s = \mathrm{id}_B
\]

This ensures we choose exactly one point in each fiber continuously. Global sections need not exist, but local sections $s: U \to E$ with $\pi \circ s = \mathrm{id}_U$ often do.

If $(U_\alpha, \varphi_\alpha)$ is a local trivialization, then such local sections always exist.

We denote the set of all (smooth) sections by:
\[
\Gamma(E) := \left\{ s: \mathcal{M} \to E \mid \pi \circ s = \mathrm{id}_{\mathcal{M}} \right\}
\]
\subsection{Minkowski Space \texorpdfstring{$\mathbb{M}^4$}{M⁴}}

The physics of this thesis takes place in Minkowski space. We therefore formally define Minkowski space $\mathbb{M}^4$.

We already defined the spacetime manifold $\mathcal{M} := \mathbb{R}^4$ in section \ref{sec:spacetime-manifold}. We now define the tangent bundle
\[
T\mathbb{M}^4 := \bigsqcup_{p \in \mathbb{M}^4} T_p \mathbb{M}^4
\]
together with the natural projection:
\[
\pi: T\mathbb{M}^4 \to \mathbb{M}^4, \quad (p, v) \mapsto p
\]

This is a trivial smooth vector bundle of rank 4 over $\mathbb{M}^4$, i.e.:
\[
T\mathbb{M}^4 \cong \mathbb{M}^4 \times \mathbb{R}^4
\]

Each fiber is:
\[
\pi^{-1}(p) \cong T_p \mathbb{M}^4 \cong \mathbb{R}^4
\]

Sections allow us to define vector fields over spacetime:
\[
s \in \Gamma(T\mathbb{M}^4)
\]

\subsubsection*{The Cotangent Space}

At this point, we introduce the cotangent space.  
The cotangent space at a point is defined as the dual of the tangent space:
\[
T_p^*\mathcal{M} := \text{Hom}_\mathbb{R}(T_p \mathcal{M}, \mathbb{R})
\]
That is, a covector $\omega \in T_p^*\mathcal{M}$ is a linear functional:
\[
\omega: T_p \mathcal{M} \to \mathbb{R}
\]

Given a coordinate basis of $T_p \mathcal{M}$:
\[
\left\{ \left. \frac{\partial}{\partial x^i} \right|_p \right\}
\]
the dual basis is:
\[
\left\{ \left. dx^i \right|_p \right\}
\]
satisfying the relation:
\[
dx^i\left( \left. \frac{\partial}{\partial x^j} \right|_p \right) = \delta^i_j
\]


This is an example of a \textbf{vector bundle}, which is a fiber bundle where the fibers are vector spaces.



\subsubsection*{The Cotangent Bundle}

Analogous to the tangent bundle, the cotangent bundle is:
\[
T^*\mathcal{M} := \bigsqcup_{p \in \mathcal{M}} T_p^* \mathcal{M}
\]
This forms a vector bundle over $\mathcal{M}$.

We define a section:
\[
\omega \in \Gamma(T^* \mathcal{M})
\]
which assigns to each $p \in \mathcal{M}$ a covector $\omega_p \in T_p^*\mathcal{M}$ smoothly. Such a section is called a \textbf{1-form}.

In a coordinate representation, a 1-form can be expressed as:
\[
\omega = \sum_{i=1}^n \omega_i(x) \, dx^i
\quad \text{with} \quad \omega_i \in C^\infty(\mathcal{M})
\]

\subsubsection*{Tensor Products and the Metric Tensor}

Since the fibers are vector spaces, we can define tensor products of bundles.

For example:
\[
T^*\mathcal{M} \otimes T^*\mathcal{M} := \bigsqcup_{p \in \mathcal{M}} T_p^*\mathcal{M} \otimes T_p^*\mathcal{M}
\]
This forms a bundle whose fibers are bilinear forms on the tangent space.

Sections of this bundle are called \((0,2)\)-tensor fields. A prominent example is the metric tensor.

We define the Minkowski metric as:
\[
\eta \in \Gamma(T^*\mathcal{M} \otimes T^*\mathcal{M})
\]

In local coordinates:
\[
\eta = \eta_{\mu\nu} \, dx^\mu \otimes dx^\nu
\quad \text{with} \quad \eta_{\mu\nu} = \text{diag}(1, -1, -1, -1)
\]

Notice that those sections are not called 2-forms. To define a \textbf{k-form}, we need to take the exterior product of two 1-forms, which is defined as follows:

Let $E \xrightarrow{\pi} B$ be a vector bundle and $\{e_\alpha\}$ be a basis of the fiber F. We define the \textbf{exterior product} as:
\[e_\alpha \wedge e_\beta \equiv e_\alpha \otimes e_\beta - e_\beta \otimes e_\alpha\]


We can define a vector bundle \( \wedge^r (E) \) of totally antisymmetric \( r \)-tensors, called the \textbf{exterior power} of the vector bundle \( E \). The fibers of \( \wedge^r (E) \) are spanned by elements of the form $\{ e_{\alpha_1} \wedge \dots \wedge e_{\alpha_r} \}$
 
The set of k-forms is defined as:
\[\Omega^r(B) \equiv \Gamma(\wedge^r (T^*B))\]



\subsubsection*{The Structure Group of a Fiber Bundle}

Above, we defined a fiber bundle as a quadruple $(E, B, \pi, F)$ equipped with local trivializations. These local trivializations are given as diffeomorphisms on an open cover of the base space $B$. Thus, the definition does not require that $U_\alpha \cap U_\beta = \emptyset$. For a point $p \in U_\alpha \cap U_\beta$, we may have multiple local trivializations $\varphi_\alpha(p, f) = \varphi_{\alpha,p}(f)$ and $\varphi_\beta(p, f) = \varphi_{\beta,p}(f)$, defined on $U_\alpha$ and $U_\beta$, respectively. 

We define the \textbf{structure group} $G$ of a fiber bundle as the Lie group of diffeomorphisms relating these local trivializations. The corresponding transition function is given by
\[
t_{\alpha\beta}(p) \equiv \varphi_{\alpha,p}^{-1} \circ \varphi_{\beta,p} : F \to F
\]

This defines a smooth map $t_{\alpha\beta}: U_\alpha \cap U_\beta \to G$ satisfying the following properties:
\begin{align*}
  t_{\alpha \alpha}(p) &= \mathrm{id}_F && \forall\, p \in U_\alpha \\
  t_{\alpha\beta}(p) &= t_{\beta\alpha}(p)^{-1} && \forall\, p \in U_\alpha \cap U_\beta \\
  t_{\alpha\beta}(p) \circ t_{\beta\gamma}(p) &= t_{\alpha\gamma}(p) && \forall\, p \in U_\alpha \cap U_\beta \cap U_\gamma
\end{align*}

In the case of the tangent bundle, the structure group is the general linear group $\mathrm{GL}(n, \mathbb{R})$, which consists of all invertible $n \times n$ matrices. This means that the transition functions between local trivializations are smooth maps from $\mathbb{R}^n$ to $\mathbb{R}^n$ that preserve linearity and invertibility.

A fiber bundle with transition maps identical to the identity map is called a \textbf{trivial bundle}. In this case, the total space $E$ is diffeomorphic to the product space $B \times F$.
In general, a fiber bundle has no unique trivialization. Let $\{\varphi_\alpha\}$ and $\{\tilde{\varphi}_\alpha\}$ be two local trivializations over the same open covering describing the same fiber bundle. Then these are related by maps $g_\alpha(p) : F \to F \quad \forall\, p \in B$ where each $g_\alpha(p)$ is a homeomorphism in the structure group $G$.
\[
g_\alpha(p) \equiv \varphi_{\alpha,p}^{-1} \circ \tilde{\varphi}_{\alpha,p}
\]
