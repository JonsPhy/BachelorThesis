
\chapter{Bundles}

  The definition of a vector on a Manifold is non-trivial because a vector space structure might not exist globally on the manifold. We can still equip a Manifold with a Vector space structure locally. Thus, tangent spaces are introduced pointwise. Combining these local structures will lead naturally to the definition of fiber bundles.

\section{Tangent Space $T_pM$}

  Let $M$ be an $n$-dimensional smooth manifold. A tangent vector at a point $p \in M$ is a linear map $v: C^\infty(\mathcal{M}) \to \mathbb{R}$ satisfying the Leibniz rule\cite{FredericSchullerDifferentialstructurespivotalconcepttangentvectorspacesLec09FredericSchuller2015}:
 \[ v[fg] = v[f]g(p) + f(p)v[g] \]

  A tangent vector at a point $p \in M$ can be constructed as the directional derivatives of an equivalence class of curves through $p$\cite{NakaharaGeometrytopologyphysics2005}.

  Let $\gamma: [-\epsilon, \epsilon] \to M$ be a smooth curve in $M$ with $\gamma(0)=p$. Then $x^\mu(\gamma(t)) \in \mathbb{R}^n$ is called the coordinate representation of $\gamma$ induces by a chart $(U, \varphi)$.

Let $f \in C^\infty(M)$ be a smooth function on $M$. The directional derivative of $f$ along the curve $\gamma$ at $t=0$ is given by:
\begin{align*}
\left. \frac{d}{dt} (f \circ \gamma)(t) \right|_{t=0}
  &= \left. \frac{d}{dt} \left( f\circ \varphi^{-1}\circ\varphi\circ\gamma(t) \right) \right|_{t=0} \\
&= \left. \frac{\partial f}{\partial x^r} \frac{d x^r}{dt} \right|_{t=0} \\
&= \left. \frac{d x^r}{dt} \right|_{t=0} \left. \frac{\partial f}{\partial x^r} \right|_p
\end{align*}

The definition of a tangent vector is now obtained by introducing an equivalence relation on curves. Two curves $\gamma_1$ and $\gamma_2$ are called equivalent at $\gamma_1(0)=\gamma_2(0)=p$ if their derivatives at $t=0$ are equal:
\[
\left. \frac{d x_1^r}{dt} \right|_{t=0}
= \left. \frac{d x_2^r}{dt} \right|_{t=0}
= v^r
\]
A tangent vector is identified with the differential operator given the equivalence class of curves. Once a chart $(U, \varphi)$ is chosen, with local coordinates $(x^1, \dots, x^n)$, a tangent vector is represented as a linear combination of partial derivatives with real coefficients.
\[
v = v^r \left. \frac{\partial}{\partial x^r} \right|_p
\]
The tangent space at a point $p \in M$ is then defined as the set of all tangent vectors at $p$ and is denoted by $T_pM$.
\[
\left\{ \left. \frac{\partial}{\partial x^1} \right|_p, \dots, \left. \frac{\partial}{\partial x^n} \right|_p \right\} \quad \text{form a basis of } T_pM
\]



\section{The Tangent Bundle as a Fiber Bundle}

To introduce the concept of fiber bundles, a detailed examination of a specific example, the tangent bundle, serves as an effective foundation.
The tangent bundle of a smooth \( n \)-dimensional manifold \( \mathcal{M} \) is constructed by taking the disjoint union of all tangent spaces \( T_p\mathcal{M} \). This construction can be represented as:
\[
T\mathcal{M} := \bigsqcup_{p \in \mathcal{M}} T_p\mathcal{M} = \bigcup_{p \in \mathcal{M}} \{p\} \times T_p\mathcal{M}
\]
Here, \( T_p\mathcal{M} \) denotes the tangent space at a point \( p \in \mathcal{M} \). The tangent space \( T_p\mathcal{M} \) is a vector space that consists of all tangent vectors at the point \( p \). It should be noted that this notation emphasizes the pointwise nature of the tangent bundle, where each tangent space is associated with a specific point on the manifold. Since each tangent space is a different object, a certain tangent vector in a tangent space can only belong to that specific tangent space. Therefore all the information is already in the vector itself and the direct product notation \( \{p\} \times T_p\mathcal{M} \) is only used to clarify\cite{FredericSchullerTopologicalmanifoldsmanifoldbundlesLec06FredericSchuller2015}.

Each element of the tangent bundle \( T\mathcal{M} \) can be expressed as a pair \( (p, v) \), where \( p \) is a point on the manifold \( \mathcal{M} \), and \( v \) is a tangent vector belonging to the tangent space \( T_p\mathcal{M} \) at that point. 

A natural projection map is defined as follows:
\[
\pi: T\mathcal{M} \to \mathcal{M}, \quad (p, v) \mapsto p
\]
This projection, \( \pi \), serves to "forget" the tangent vector \( v \) associated with each point, effectively collapsing all the tangent vectors at \( p \) to the single point \( p \) in the base manifold\cite{NakaharaGeometrytopologyphysics2005}.

The fiber over a point \( p \) is denoted as \( \pi^{-1}(p) = T_p\mathcal{M} \). This represents all the tangent vectors at the point \( p \). Since \( T_p\mathcal{M} \) is isomorphic to \( \mathbb{R}^n \) as a vector space, it is referred to as the model fiber, denoted \( F = \mathbb{R}^n \)\cite{NakaharaGeometrytopologyphysics2005}.

Consider a coordinate chart \( (U, \varphi) \) on \( \mathcal{M} \). The chart \( \varphi \) provides a mapping from an open set \( U \subseteq \mathcal{M} \) to an open subset of \( \mathbb{R}^n \). A diffeomorphism on the preimage \( \pi^{-1}(U) \subset T\mathcal{M} \) can then be defined as:
\[
\Psi: \pi^{-1}(U) \to \mathbb{R}^{2n}, \quad (p, v) \mapsto \left(x^1(p), \dots, x^n(p), v^1, \dots, v^n \right)
\]
where the coordinates \( x^i(p) \) represent the local coordinates of the point \( p \) in \( U \), and \( v = v^i \left. \frac{\partial}{\partial x^i} \right|_p \) describes the tangent vector in terms of its components in the chosen coordinate system\cite{NakaharaGeometrytopologyphysics2005}.

This establishes a local trivialization of the tangent bundle, expressed as:
\[
TM|_U \cong U \times \mathbb{R}^n
\]

In summary, the construction of the tangent bundle yields a fiber bundle characterized by the following essential components:
\begin{itemize}
  \item \textbf{Total space:} \( T\mathcal{M} = \bigsqcup_{p \in \mathcal{M}} T_p\mathcal{M} \), encapsulating all tangent spaces.
  \item \textbf{Base space:} \( \mathcal{M} \), a manifold to which additional structure is added.
  \item \textbf{Projection:} \( \pi: T\mathcal{M} \to \mathcal{M} \), mapping each tangent vector to its associated point on the manifold.
  \item \textbf{Model fiber:} \( F = \mathbb{R}^n \), serving as the standard fiber structure over each point.
  \item \textbf{Local trivialization:} \( T\mathcal{M}|_U \cong U \times \mathbb{R}^n \), ensuring that the tangent bundle locally resembles a product structure.
\end{itemize}

 Similar to the way a manifold is commonly perceived as a space that locally resembles \( \mathbb{R}^n \), a fiber bundle may be conceptualized as a space that locally resembles the Cartesian product of the base space with a typical fiber structure. 

Since the fiber of the tangent bundle is a vector space, the tangent bundle is also referred to as a \textbf{vector bundle}.

\section{Definition of a Fiber Bundle}

The formal definition of a fiber bundle reads as follows\cite{FredericSchullerTopologicalmanifoldsmanifoldbundlesLec06FredericSchuller2015}.
A fiber bundle is a quadruple $(E, B, \pi, F)$ where:
\begin{itemize}
  \item $E$ is the total space
  \item $B$ is the base space
  \item $\pi: E \to B$ is a surjective map called the projection
  \item $F$ is the typical fiber
\end{itemize}

There exists an open cover $\{U_\alpha\}$ of $B$ such that for each $\alpha$, there is a diffeomorphism
\[
\varphi_\alpha: \pi^{-1}(U_\alpha) \to U_\alpha \times F
\]
such that the following diagram commutes:


\begin{center}
\begin{tikzpicture}[scale=1.5, every node/.style={font=\normalsize}]
  \node (A) at (0,1.5) {$\pi^{-1}(U)$};
  \node (B) at (4,1.5) {$U \times F$};
  \node (C) at (0,0) {$U$};

  \draw[->] (A) -- (B) node[midway, above] {$\varphi$};
  \draw[->] (A) -- (C) node[midway, left] {$\pi$};
  \draw[->] (B) -- (C) node[midway, right] {$\mathrm{proj}_1$};
\end{tikzpicture}
\end{center}


where $\text{proj}_1: A \times B \to A$ is the first projection.

$(U_\alpha, \varphi_\alpha)$ are called \emph{local trivialization}.

The fiber over a point $b \in B$ is:
\[
F_b := \pi^{-1}(\{b\}) \cong F
\]

Often the notations $E \xrightarrow{\pi} B$ or $\pi : E \to B$ are used to denote a fiber bundle.


\subsection{The Structure Group of a Fiber Bundle}

Above a fiber bundle was defined as a quadruple \((E, B, \pi, F)\) equipped with local trivializations. These local trivializations are established as diffeomorphisms on an open cover $\{U_\alpha\}$ of the base space \(B\). The definition does not impose the requirement that \(U_\alpha \cap U_\beta = \emptyset\). For a point \(p \in U_\alpha \cap U_\beta\), multiple local trivializations \(\varphi_\alpha(p, f) = \varphi_{\alpha,p}(f)\) and \(\varphi_\beta(p, f) = \varphi_{\beta,p}(f)\) may be present, defined on \(U_\alpha\) and \(U_\beta\), respectively.

The \textbf{structure group} \(G\) of a fiber bundle is defined as the Lie group of diffeomorphisms relating these local trivializations. The corresponding transition function is given by\cite{NakaharaGeometrytopologyphysics2005}:

\[
t_{\alpha\beta}(p) \equiv \varphi_{\alpha,p}^{-1} \circ \varphi_{\beta,p} : F \to F
\]

This establishes a smooth map \(t_{\alpha\beta}: U_\alpha \cap U_\beta \to G\) that satisfies the following properties:

\begin{align*}
  t_{\alpha \alpha}(p) &= \mathrm{id}_F && \forall\, p \in U_\alpha \\
  t_{\alpha\beta}(p) &= t_{\beta\alpha}(p)^{-1} && \forall\, p \in U_\alpha \cap U_\beta \\
  t_{\alpha\beta}(p) \circ t_{\beta\gamma}(p) &= t_{\alpha\gamma}(p) && \forall\, p \in U_\alpha \cap U_\beta \cap U_\gamma
\end{align*}

In the case of the tangent bundle, the structure group corresponds to the general linear group \(\mathrm{GL}(n, \mathbb{R})\), which consists of all invertible \(n \times n\) matrices. 
A fiber bundle with transition maps identical to the identity map is termed a \textbf{trivial bundle}. In this scenario, the total space \(E\) is diffeomorphic to the product space \(B \times F\). 

Generally, a fiber bundle does not possess a unique trivialization. Let \(\{\varphi_\alpha\}\) and \(\{\tilde{\varphi}_\alpha\}\) denote two local trivializations over the same open covering that describe the same fiber bundle. These trivializations are related by maps \(g_\alpha(p) : F \to F \quad \forall\, p \in B\), where each \(g_\alpha(p)\) is a homeomorphism within the structure group \(G\). It will be defined in section \ref{sec:action-of-lie-group-on-manifold} what it means for a Lie group to act on a manifold. The transition function between the two local trivializations is then given by:

\[
g_\alpha(p) \equiv \varphi_{\alpha,p}^{-1} \circ \tilde{\varphi}_{\alpha,p}
\]

Considering the tangent bundle as an illustrative example\cite{NakaharaGeometrytopologyphysics2005}, let \(U_i\) and \(U_j\) represent overlapping charts with \(p \in U_i \cap U_j\). Utilizing the basis \(\left\{ \left. \frac{\partial}{\partial x^i} \right|_p \right\}\) and \(\left\{ \left. \frac{\partial}{\partial y^j} \right|_p \right\}\), a vector \(v \in T_p\mathcal{M}\) can be expressed in both bases as:

\[
v = v^{\mu} \frac{\partial}{\partial x^{\mu}} = \tilde{v}^{\mu} \frac{\partial}{\partial y^{\mu}}
\]

The transition function \(t^\nu_{\,\,\mu}\) is thus defined as:

\[
\tilde{v}^\nu = \left.\frac{\partial y^\nu}{\partial x^\mu}\right|_p v^\mu = t^\nu_{\,\,\mu} v^\mu
\]

\subsection{Sections}

An important definition in the context of fiber bundles is that of a section or cross-section. This concept enables the selection of an element from each fiber over each point in a continuous manner, facilitating the introduction of ideas such as vector fields over spacetime\cite{NakaharaGeometrytopologyphysics2005}.

A \textbf{section} of a fiber bundle \(\pi : E \to B\) is defined as a continuous map \(s: B \to E\) such that

\[
\pi \circ s = \mathrm{id}_B.
\]

This condition ensures that exactly one point is chosen from each fiber continuously. 
The set of all (smooth) sections is denoted by:
\[
\Gamma(E) := \left\{ s: \mathcal{M} \to E \mid \pi \circ s = \mathrm{id}_{\mathcal{M}} \right\}.
\]
It is also possible to define a section locally on an open set \(U \subset B\) as a map \(s: U \to E\) such that \(\pi \circ s = \mathrm{id}_U\). In this case, the section is called a \textbf{local section}.


\section{The cotangent bundle and differential forms}


At this point, the cotangent space is introduced. The cotangent space at a point is defined as the dual of the tangent space:
\[
T_p^*\mathcal{M} := \text{Hom}_\mathbb{R}(T_p \mathcal{M}, \mathbb{R})
\]
A covector \(\omega \in T_p^*\mathcal{M}\) is a linear functional:
\[
\omega: T_p \mathcal{M} \to \mathbb{R}
\]

Given a coordinate basis of \(T_p \mathcal{M}\):
\[
\left\{ \left. \frac{\partial}{\partial x^i} \right|_p \right\}
\]
the dual basis is:
\[
\left\{ \left. dx^i \right|_p \right\}
\]
satisfying the relation:
\[
dx^i\left( \left. \frac{\partial}{\partial x^j} \right|_p \right) = \delta^i_j
\]
\subsubsection*{The Cotangent Bundle}

Analogous to the tangent bundle, the cotangent bundle is defined as:
\[
T^*\mathcal{M} := \bigsqcup_{p \in \mathcal{M}} T_p^* \mathcal{M}
\]
This structure constitutes a vector bundle over \(\mathcal{M}\).

A section of the cotangent bundle can be defined as:
\[
\omega \in \Gamma(T^* \mathcal{M})
\]
This section assigns to each \(p \in \mathcal{M}\) a covector \(\omega_p \in T_p^*\mathcal{M}\) smoothly. Such a section is referred to as a \textbf{1-form}.

In a coordinate representation, a 1-form can be expressed as:
\[
\omega = \sum_{i=1}^n \omega_i(x) \, dx^i
\quad \text{with} \quad \omega_i \in C^\infty(\mathcal{M})
\]

\subsubsection*{Tensor Products and the Metric Tensor}

Utilizing the fact that the fibers are vector spaces, tensor products of bundles can be defined.

For instance:
\[
T^*\mathcal{M} \otimes T^*\mathcal{M} := \bigsqcup_{p \in \mathcal{M}} T_p^*\mathcal{M} \otimes T_p^*\mathcal{M}
\]
This forms a bundle whose fibers consist of bilinear forms on the tangent space.

Sections of this bundle are referred to as \((0,2)\)-tensor fields. A prominent example of such fields is the metric tensor.

The Minkowski metric is defined as:
\[
\eta \in \Gamma(T^*\mathcal{M} \otimes T^*\mathcal{M})
\]

In local coordinates, the Minkowski metric can be expressed as:
\[
\eta = \eta_{\mu\nu} \, dx^\mu \otimes dx^\nu
\quad \text{with} \quad \eta_{\mu\nu} = \text{diag}(1, -1, -1, -1)
\]

It is important to note that these sections are not classified as 2-forms. To define a \textbf{k-form}, it is necessary to take the exterior product of two 1-forms, which is defined as follows:

Let \(E \xrightarrow{\pi} B\) be a vector bundle and \(\{e_\alpha\}\) be a basis of the fiber \(F\). The \textbf{exterior product} is defined by:
\[e_\alpha \wedge e_\beta \equiv e_\alpha \otimes e_\beta - e_\beta \otimes e_\alpha\]

A vector bundle \(\wedge^r (E)\) of totally antisymmetric \(r\)-tensors, known as the \textbf{exterior power} of the vector bundle \(E\), can be defined. The fibers of \(\wedge^r (E)\) are spanned by elements of the form \(\{ e_{\alpha_1} \wedge \dots \wedge e_{\alpha_r} \}\).

The set of k-forms is defined as:
\[\Omega^r(B) \equiv \Gamma(\wedge^r (T^*B))\]

The \textbf{exterior derivative} is then defined as a map \cite{NakaharaGeometrytopologyphysics2005}:

\begin{align*}
  d: \Omega^r(B) &\to \Omega^{r+1}(B) \\
  \omega &\mapsto d\omega = \frac{1}{r!}\left( \frac{\partial}{ \partial x^\nu} \omega_{\mu_1 \dots \mu_r} \right) dx^\nu \wedge dx^{\mu_1} \wedge \dots \wedge dx^{\mu_r}
\end{align*}
