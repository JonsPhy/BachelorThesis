\chapter{Introduction}

Differential geometry plays a central role in the formulation of modern physics. Many physical theories, notably gauge theories and general relativity, are naturally expressed on smooth manifolds equipped with geometric structures such as tensors, connections, and curvature. These mathematical tools provide a coordinate-independent framework in which physical laws can be formulated and understood. In the 20th century Wu and Yang formalized the idea that gauge fields, such as those describing electromagnetism or the strong and weak nuclear forces, can be understood as connections on principal bundles\cite{WuConceptnonintegrablephasefactorsglobalformulationgaugefields1975}, building on the work of Charles Ehreshmann. In this context, the curvature of a connection corresponds directly to the physical field strength, while gauge transformations are interpreted as changes of local trivialization. This geometric formulation reveals the profound mathematical structure behind gauge invariance and highlights the important role of symmetry in physics. The goal of this thesis is to elucidate how such geometric structures naturally lead to gauge fields. Starting from the definition of differentiable manifolds, more refined concepts are introduced—vector bundles, principal bundles, and connections—culminating in a complete formulation of classical gauge theory within the framework of differential geometry. Throughout, the focus remains on building the theory in a mathematically consistent manner. 

\textbf{Structure}

This thesis develops the geometric foundations of gauge theories by tracing a path from differentiable manifolds to fiber bundles, connections, and curvature. Starting with manifolds, tangent spaces, and bundles, the necessary tools from topology and geometry are introduced. Building on this, principal bundles and their connections are defined, leading to a geometric formulation of gauge fields as connection forms and their curvature as the field strength. Finally, classical gauge theories used in standard model physics are reintroduced within this framework, illustrating the reinterpretation of gauge fields as geometric objects.

\textbf{Objective}

The primary objective is to demonstrate how the mathematical concept of a connection on a principal bundle provides a natural and rigorous formulation of gauge fields. This approach underscores that many features of physical theories, such as field strength and gauge invariance, are consequences of the underlying geometric structure.
This thesis is concerned with the classical, differential-geometric formulation of gauge fields. Quantum aspects, topological phenomena, and the coupling of gauge fields to matter are not treated. The emphasis lies on comprehending the mathematical structure underlying gauge invariance and field strength in a coordinate-free way from a purely geometric perspective. Also, the theory of Lie groups will not be introduced, still some definitions will be provided. Underlying concepts can be found in sources used throughout this theses such as \cite{RaghunathanLieGroupsLieAlgebras2025} and \cite{NakaharaGeometrytopologyphysics2005}
