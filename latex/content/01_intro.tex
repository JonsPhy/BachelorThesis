\chapter{Introduction}

Differential geometry plays a central role in the formulation of modern physics. Many physical theories, notably gauge theories and general relativity, are naturally expressed on smooth manifolds equipped with geometric structures such as tensors, connections, and curvature. These mathematical tools provide a coordinate-independent framework in which physical laws can be formulated and understood. In the 20th century Wu and Yang formalized the idea that gauge fields, such as those describing electromagnetism or the strong and weak nuclear forces, can be understood as connections on principal bundles\cite{WuConceptnonintegrablephasefactorsglobalformulationgaugefields1975}. In this context, the curvature of a connection corresponds directly to the physical field strength, while gauge transformations are interpreted as changes of local trivialization. This geometric formulation reveals the profound mathematical structure behind gauge invariance and highlights the important role of symmetry in physics. The goal of this thesis is to elucidate how such geometric structures naturally lead to gauge fields. Starting from the definition of differentiable manifolds, more refined concepts are introduced—vector bundles, principal bundles, and connections—culminating in a complete formulation of classical gauge theory within the framework of differential geometry. Throughout, the focus remains on building the theory in a mathematically consistent manner. 

\textbf{Structure of the Thesis}

Chapter 2 gives preliminary concepts from topology and introduces the basic definitions of differentiable manifolds. Chapter 3 presents fiber bundles, beginning with the tangent bundle as a motivating example. The general definition of fiber bundles is elaborated, including the concepts of structure group, sections, and local trivializations. The cotangent bundle and differential forms are discussed to establish the necessary tools for field theories. Chapter 4 focuses on principal bundles, which utilize a Lie group as the typical fiber. The frame bundle is explored as a key example, and the connection to symmetry groups is elucidated. Chapter 5 introduces connections on principal bundles. Both the global connection one-form and the local connection form (interpreted as a gauge potential) are defined, and the notion of horizontal and vertical subspaces is introduced. The transformation behavior under gauge transformations is derived. Chapter 6 defines the curvature of a connection and demonstrates how it leads to the field strength in physical gauge theories. The Bianchi identity, a fundamental identity satisfied by the curvature, is derived both globally and locally. Chapter 7 applies the developed framework to classical gauge theories, beginning with Maxwell theory as a U(1) gauge theory and subsequently extending the discussion to the general structure of Yang–Mills theories. Chapter 8 summarizes the results and briefly discusses possible extensions, including the inclusion of matter fields or spontaneous symmetry breaking. 

\textbf{Scope and Approach}

This thesis is concerned primarily with the classical, differential-geometric formulation of gauge fields. All manifolds and bundles are assumed to be smooth and locally trivial. Quantum aspects, topological phenomena, and the coupling of gauge fields to matter are not treated. The emphasis lies on comprehending the mathematical structure underlying gauge invariance and field strength from a purely geometric perspective. 

\textbf{Objective}

The primary objective is to demonstrate how the mathematical concept of a connection on a principal bundle provides a natural and rigorous formulation of gauge fields. This approach underscores that many features of physical theories, such as field strength and gauge invariance, are consequences of the underlying geometric structure.
