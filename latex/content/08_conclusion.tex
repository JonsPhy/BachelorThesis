
\chapter{Conclusion}

The above chapters established the foundations of modern physics in a geometrical framework. Chapter 2 presented preliminary concepts from topology and introduced the basic definitions of differentiable manifolds. Chapter 3 discussed fiber bundles, beginning with the tangent bundle as a motivating example. The general definition of fiber bundles was elaborated, including structure groups, sections, and local trivializations. The cotangent bundle and differential forms were examined to establish the necessary tools for field theories. Chapter 4 focused on principal bundles, which utilized a Lie group as the typical fiber. The frame bundle was explored as a key example, and the connection to symmetry groups was elucidated. Chapter 5 introduced connections on principal bundles. The global connection one-form and the local connection form (interpreted as a gauge potential) were defined, and horizontal and vertical subspaces were introduced. The transformation behavior under gauge transformations was derived. Chapter 6 defined the curvature of a connection and demonstrated how it led to the field strength in physical gauge theories. The Bianchi identity, a fundamental identity satisfied by the curvature, was derived both globally and locally. Chapter 7 applied the developed framework to classical gauge theories, beginning with Maxwell theory as a U(1) gauge theory and extending the discussion to the general structure of Yang–Mills theories. This journey—from manifolds to fiber bundles and curvature—has shown how the language of differential geometry offers not just a reformulation, but a profound reinterpretation of gauge theories as geometric objects.

This thesis focused on the mathematical foundations of gauge theories, which are an important aspect in physics, especially in the context of quantum field theories. Another essential subtopic in the field of differential geometry in physics is the study of associated bundles, which was not covered in this thesis. Associated bundles are bundles that transform under a representation of gauge group. This is of interest, particularly in relation to the standard model of particle physics, where associated bundles are used to describe matter fields.
\\
Furthermore, this introduced framework can be applied to explore the Higgs mechanism, which is a crucial aspect of the standard model of particle physics. The Higgs mechanism involves spontaneous symmetry breaking and the generation of masses. In the context of differential geometry, this can be understood as a reduction of the symmetry group associated with the principal bundle. This reduction can be analyzed using the tools developed in this thesis, providing a deeper understanding of how gauge theories can incorporate mass generation through geometric structures.
\\
The geometric approach not only clarifies the structure of gauge theories, but also reveals profound mathematical coherence—suggesting that the language of geometry is the natural framework for modern physics.
