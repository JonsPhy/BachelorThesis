
\chapter{Connections on Principal Bundles}

\section{General Definition}

A Connection is a consistent way to separate the tangent space of a principal bundle into a vertical subspace tangent to the fiber and a horizontal subspace that complements the former. This will be constructed by considering a vector field generated by the right action of the one-parameter subgroup of the structure group $G$ on the principal bundle. As before a vector is defined by its action on a function evaluated along a curve which will be generated by the former mentioned right action.

Let \( P \xrightarrow{\pi} M \) be a principal bundle with structure group \( G \). The right action of \( G \) on \( P \) induces a vector field as follows: For each \( A \in \mathfrak{g} \cong T_eG \), the action of the one-parameter subgroup \( \exp(tA) \) on an element \( p \in P \) yields a curve. Since the group acts within the fiber, it holds that \( \pi(p) = \pi(p \triangleleft \exp(tA)) = p \). A vector \( X^A_p \in T_pP \) is defined by its action on a function \( f \in C^\infty(P) \) \cite{NakaharaGeometrytopologyphysics2005}:

\[
X^A_p f = \frac{d}{dt} f(p \triangleleft \exp(tA))\mid_{t=0}.
\]

Furthermore, a vector space isomorphism \( i: \mathfrak{g} \longrightarrow \Gamma(TP) \) is defined that assigns to each element \( A \in \mathfrak{g} \) the vector field \( X^A \) which is generated by $A$. This vector field is referred to as a \textbf{fundamental vector field} on \( P \).

The \textbf{Pushforward} \cite{Pushforward2025} of a smooth map \( F:M \longrightarrow N \) between smooth manifolds \( M \) and \( N \) is defined as a map between the tangent spaces:

\[
F_* : T_pM \longrightarrow T_{F(p)}N.
\]

This is identified using \( (F_*v)(f) = v(f\circ F) \) for \( v \in T_pM \) and \( f \in C^\infty(N) \).

The pushforward of the projection map \( \pi_* : TP \longrightarrow TM \) facilitates the construction of the \textbf{vertical subspace} \( V_pP \coloneq \ker(\pi_*) \) at a point \( p \in P \), which serves as a vector subspace of the tangent space of \( P \).
It is noteworthy that each fundamental vector \( X^A_p \in V_pP \), given that \( \pi_* (X^A_p) = 0 \) by construction.

A \textbf{connection} on a principal bundle \( P \xrightarrow{\pi} M \) is defined as a decomposition of the tangent space \( T_pP \) into a vertical subspace \( V_pP \) and a \textbf{horizontal subspace} \( H_pP \). This is achieved by choosing a complement to the vertical subspace at each point \( p \in P \) such that\cite{DudekEhreshmanntheoryconnectionprincipalbundlecompendiumphysicists2018}:

\begin{align*}
  \text{(i)}\quad 
    & T_pP = H_pP \oplus V_pP \\
  \text{(ii)}\quad 
    & (\triangleleft g)_* (H_pP) = H_{p \triangleleft g}P 
    \quad \text{for all } g \in G \\
  \text{(iii)}\quad 
    & \text{For every smooth vector field } X \in \Gamma(TP), \text{ the unique decomposition } 
      X = X^H + X^V \\
    & \text{with } X^H(p) \in H_pP \text{ and } X^V(p) \in V_pP \text{ produces smooth vector fields } 
      X^H \in \Gamma(HP),\ X^V \in \Gamma(VP).
\end{align*}

Condition (ii) ensures that when moving along the fibers via the action of \( G \), the horizontal subspace changes smoothly, while condition (iii) guarantees that the horizontal subspace varies smoothly when traversing the manifold \( P \).


\section{Connection One-Form}

The choice of a horizontal subspace at each point \( p \in P \) can be achieved by defining a Lie algebra-valued one-form. The horizontal subspace is then interpreted as the kernel of this one-form. The \textbf{connection one-form} \( \omega \in \mathfrak{g} \otimes T^*P \) is defined as a \( \mathfrak{g} \)-valued one-form on \( P \) such that:

\begin{align*}
  \text{(i)}\quad 
    & \omega(X^A) = A \quad \text{for all } A \in \mathfrak{g} \\
  \text{(ii)}\quad 
    & (\triangleleft g)^* \omega = \text{Ad}_{g^{-1*}} \omega \quad \text{for all } g \in G 
\end{align*}

Here, \( (\triangleleft g)^* \omega \) denotes the pullback of the connection one-form by the right action of \( g \in G \) on \( P \), and \( \text{Ad}_{g^{-1}} \) is the adjoint action of \( g^{-1} \) on the Lie Group \( G \). This implies that \( \omega_{p \triangleleft g}(X_p(\triangleleft g)_*) = g^{-1} \cdot_* \omega_p(X_p) \cdot_* g \). The horizontal subspace \( H_pP \) is then defined as the kernel of the connection one-form\cite{NakaharaGeometrytopologyphysics2005}
\[ H_pP \equiv \{ X \in T_pP \mid \omega(X)=0 \} \]

This is consistent with the general definition of a connection. The smoothness of the decomposition is guaranteed by the fact that the connection one-form is a section, which is smooth by definition. It is therefore sufficient to show that the horizontal subspace is invariant under the right action of \( G \).
To show this, consider a vector $X$ at a point $p \in P$ such that $X \in H_pP$. By definition $\omega(X) = 0$. For any element $g \in G$ the pushforward of the right action on $X$ can be acted on by $\omega$\cite{NakaharaGeometrytopologyphysics2005}:
\[ \omega((\triangleleft g)_* X) = (\triangleleft g)^* \omega(X) = g^{-1} \cdot_* \omega(X) \cdot_* g = 0 \]
Therefore $(\triangleleft g)^* X$ is again a horizental vector at the point $p\triangleleft g$. Futhermore, any vector $\tilde{X} \in H_{p\triangleleft g}P$ can thus be obtained by the right action on some vector $X \in H_pP$. Therefore $ (\triangleleft g)_* (H_pP) = H_{p \triangleleft g}P $


\section{Local Connection Form}

The connection one-form, as defined above, is a global object on the principal bundle \( P \). However, in practice, it is often useful to work with local connection forms, which will be identified with the gauge potential in physical gauge theories.

Consider an open covering \( \{U_i\} \) of the base manifold \( M \) and local sections \( \sigma_i : U_i \rightarrow \pi^{-1}(U_i) \). A Lie-algebra valued one-form \( \mathcal{A}_i \equiv \sigma_i^* \omega \in \mathfrak{g} \otimes\Omega^1(U_i) \) is defined for a global connection one-form \( \omega \) \cite{NakaharaGeometrytopologyphysics2005}. This local connection form is termed a \textbf{Yang-Mills field} \cite{FredericSchullerLocalrepresentationsconnectionbasemanifoldYangMillsfieldsLec222015}.

Given a local section \( \sigma_i : U_i \to P \), a local trivialization is established:
\begin{align*}
  \psi_i : U_i \times G &\longrightarrow \pi^{-1}(U_i) \subset P \\
  (p, g) &\mapsto \sigma_i(p) \triangleleft g
\end{align*}

This trivialization introduces a local representation of the global connection one-form \( \omega \) via its pullback:

\begin{align*}
  \psi_i^* \omega : T_{(p, g)}(U_i \times G) &\longrightarrow \mathfrak{g} \\
  (\psi_i^* \omega)_{(p, g)}(X) &= \omega_{\sigma_i(p) \triangleleft g}\left((\psi_i)_* X\right)
\end{align*}

The relations of the above maps are illustrated in the following diagram:
\begin{figure}[h!]
  \centering
  \begin{tikzpicture}[scale=1.3, every node/.style={scale=1}]
    % Matrix for main layout
    \matrix (m) [matrix of math nodes, row sep=3.7em, column sep=4.7em, ampersand replacement=\&] {
    \& \& \\
    \& U_i \times G \& \pi^{-1}(U_i) \subset P \\
    \& U_i \& M \\
    \& \& \\
    };

    % Manually placed nodes for forms
    \node (omegaPsi) at ($(m-2-2) + (-1.2, 1)$) {\(\psi_i^* \omega\)};
    \node (omegaSigma) at ($(m-3-2) + (-1.2, -1)$) {\(\sigma_i^* \omega\)};

    % Solid arrows (maps)
    \path[->]
    (m-2-2) edge node[above] {\(\psi_i\)} (m-2-3)
    (m-3-2) edge node[left] {\(\sigma_i\)} (m-2-3)
    (m-3-2) edge node[below] {\(\text{id}\)} (m-3-3)
    (m-2-2) edge node[left] {\(\text{proj}_1\)} (m-3-2)
    (m-2-3) edge node[right] {\(\pi\)} (m-3-3);

    % Dashed arrows for associated forms
    \path[dashed,->]
    (omegaPsi) edge[out=-60, in=120] (m-2-2)
    (omegaSigma) edge[out=60, in=-120] (m-3-2);

  \end{tikzpicture}
  \caption{Local trivialization \(\psi_i\) and local connection forms \(\psi_i^*\omega\) and \(\sigma_i^*\omega\) associated to \(U_i \times G\) and \(U_i\), respectively.}
\end{figure}

This local representation is related to the Yang-Mills field \( \mathcal{A}_i \) by \cite{FredericSchullerLocalrepresentationsconnectionbasemanifoldYangMillsfieldsLec222015}:

\begin{align*}
  (\psi_i^* \omega)_{(p, g)}(X) &= \text{Ad}_{g^{-1*}} \left(\mathcal{A}_i (X)\right) + \Xi_g(X) \\
\end{align*}

Here, \( \Xi \) denotes the \textbf{Maurer–Cartan form} of the Lie group \( G \). This form takes a tangent vector \( v \in T_gG \) and maps it to the unique Lie algebra element (i.e., a tangent vector at the identity) that generates \( v \) via left translation:

\[
\Xi(v) = (g^{-1} \triangleright)_* v \in T_eG \cong \mathfrak{g}
\]

This formulation exploits the fact that every tangent vector on \( G \) arises as the pushforward of a unique element of the Lie algebra \( \mathfrak{g} = T_eG \) under left action\cite{RaghunathanLieGroupsLieAlgebras2025}. Thus, for every \( v \in T_gG \), there exists a unique \( X \in \mathfrak{g} \) such that

\[
v = (g \triangleright)_* X.
\]

Consequently, \( \Xi \) identifies the tangent bundle \( TG \) with \( G \times \mathfrak{g} \) via left translation \cite{MaurerCartanform2025}.

\section{Connection on the Frame Bundle}

The Frame Bundle $LM$ is of particular interest, because many groups relevant in physics are subgroups of the general linear group \( GL(n, \mathbb{R}) \) or \( GL(n, \mathbb{C}) \). Therefore in the following a local connection form and the Maurer-Cartan form will be derived.

Any choice of a chart \( (U_i, x) \) on the base manifold \( M \) induces a section on the frame bundle \( LM \) by associating to each point \( m \in U_i \) the frame given by its coordinates. This section is denoted as:

\begin{align*}
  \sigma : U_i &\longrightarrow LM \\
  m &\mapsto \sigma_i(m) \coloneq \left( \left. \frac{\partial}{\partial x^1} \right|_m, \dots , \left. \frac{\partial}{\partial x^{\text{dim}M}} \right|_m  \right)
\end{align*}



Then the Yang-Mills field \( \mathcal{A}_i = \sigma_i^* \omega \) is a one-form on \( U_i \) with values in the Lie algebra $\mathfrak{gl}(\text{dim}M,\mathbb{R}) = \{ M \mid M \text{ is a } n\times n \text{ matrix with components } M^\alpha_{\,\beta}\in \mathbb{R} \}$. The Yang-Mills field can be expressed in its components as:
\[ (\mathcal{A}^i)^\alpha_{\,\,\beta\mu} \]

Where $\alpha, \,  \beta$ are labels for the Lie algebra components and $\mu$ is the index of the base manifold. 
The Maurer-Cartan form \( \Xi \) can be constructed as follows:

Let \( gl \subseteq GL(d,\mathbb{R}) \) be an open subset of the general linear group containing the identity. Coordinates are introduced by:
\begin{align*}
  x: gl &\longrightarrow \mathbb{R} \\
  g &\mapsto x(g)^a_{\,\,b} \coloneq g^a_{\,\,b}
\end{align*}

Consider a left-invariant vector field \( L^A \) generated by the Lie algebra element \( A \in \mathfrak{gl}(d, \mathbb{R}) \). Since it is a vector field on the group, it acts on the coordinate functions:

\begin{align*}
  \left( L^A x^a_{\,\,b} \right)_g &= x^a_{\,\,b} \frac{d}{dt} \left( g \cdot \exp(tA) \right) \bigg|_{t=0} \\
  &= \frac{d}{dt} \left( g^a_{\,\,c} \exp(tA)^c_{\,\,b} \right) \bigg|_{t=0} \\
  &= g^a_{\,\,c} A^c_{\,\,b}
\end{align*}


Therefore the components of the vector field are given by \( L^A_g = g^a_{\,\,b} \, A^b_{\,\,c} \, \frac{\partial}{\partial x^a_{\,\,c}} \)\cite{FredericSchullerLocalrepresentationsconnectionbasemanifoldYangMillsfieldsLec222015}

The Maurer-Cartan form $\Xi$ then is defined as the one-form that maps the left-invariant vector field \( L^A \) to the Lie algebra element \( A \):
\[ (\Xi_g)^a_{\,\,b} = (g^{-1})^a_{\,\,c}(dx^c_{\,\,b}) \]

It can be easily checked that this expression satisfies the properties of a Maurer-Cartan form:

\begin{align*}
  \Xi_g(L^A_g) 
  &= (g^{-1})^a_{\,c} \, (dx)^c_{\,b} \left( g^p_{\,r} \, A^r_{\,q} \, \frac{\partial}{\partial x^p_{\,q}} \right) \\
  &= (g^{-1})^a_{\,c} \, g^p_{\,r} \, A^r_{\,q} \left( (dx)^c_{\,b} \, \frac{\partial}{\partial x^p_{\,q}} \right) \\
  &= (g^{-1})^a_{\,c} \, g^p_{\,r} \, A^r_{\,q} \, \delta^c_{\,p} \, \delta^q_{\,b} \\
  &= (g^{-1})^a_{\,p} \, g^p_{\,r} \, A^r_{\,b} \\
  &= A^a_{\,b}
\end{align*}


\section{Compatibility condition for local connection forms}


It was stated before, that the local connection forms $\mathcal{A}_i$ relate to a unique global connection one-form $\omega$. For this to be true, the local connection forms must satisfy a compatibility condition. This condition is given by the requirement that the local connection forms on overlapping charts \( U_i \cap U_j \neq \emptyset\) are related by a gauge transformation\cite{NakaharaGeometrytopologyphysics2005}. Specifically, let $\sigma_i$ and $\sigma_j$ be sections respectively defining Yang-Mills fields \( \mathcal{A}_i \) and \( \mathcal{A}_j \) on the overlapping region \( U_i \cap U_j \). Introduce a gauge map
\[ \Omega : U_i \cap U_j \longrightarrow G \]
defined by the relation
\[ \sigma_j(m) = \sigma_i(m) \triangleleft \Omega(m) \quad \forall m \in U_i \cap U_j \]


Then the local connection forms are related as follows:

\[ \mathcal{A}_j = \text{Ad}_{\Omega^{-1}(m)*} \mathcal{A}_i + \Omega^*\Xi_m \]

In this will be shown for the case of the frame bundle \( LM \). First, we calculate the latter expression. Notice that \( \Omega^* \Xi_m \) is a map from the tangent space of the intersection on the base manifold \( U_i \cap U_j \) to the Lie algebra \( \mathfrak{gl}(d, \mathbb{F}) \) for $\mathbb{F} = \mathbb{R,C}$. Therefore, to find the explicit form, it is calculated how this map acts on a vector in the tangent space:
\begin{align*}
  (\Omega^* \Xi)_p \left( \frac{\partial}{\partial x^\mu} \right)_p 
  &= \Xi_{\Omega(p)}\left( \left( \Omega_* \left( \frac{\partial}{\partial x^\mu} \right)_p \right)_{\Omega(p)} \right) \\
  &= (\Omega^{-1}(p))^i_{\,k}(dx^k_{\,j})_{\Omega(p)}\left( \left( \Omega_* \left( \frac{\partial}{\partial x^\mu} \right)_p \right)_{\Omega(p)} \right) \\
  &= \Omega^{-1}(p)^i_{\,k} \left( \Omega_* \left( \frac{\partial}{\partial x^\mu} \right)_p \right)_{\Omega(p)} \left( x^k_{\,j} \right) \\
  &= \Omega^{-1}(p)^i_{\,k} \left( \frac{\partial}{\partial x^\mu}  \right)_p \left( x^k_{\,j} \circ \Omega \right)_p \\
  &= \Omega^{-1}(p)^i_{\,k} \left( \frac{\partial}{\partial x^\mu}  \right)_p \Omega(p)^k_{\,j}
\end{align*}

Therefore, the components of the pullback of the Maurer–Cartan form are given by\cite{FredericSchullerLocalrepresentationsconnectionbasemanifoldYangMillsfieldsLec222015}:
\[
\left( (\Omega^* \Xi)_p \right)^i_{\,j} 
= \Omega^{-1}(p)^i_{\,k} \left( \frac{\partial}{\partial x^\mu} \right)_p \Omega(p)^k_{\,j} \, dx^\mu 
\coloneqq \mathbf{\Omega}^{-1} d\mathbf{\Omega}
\]



Futhermore, the pushforward of the adjoint action on the Yang-Mills field is easily obtained by definition of the adjoint action:
\begin{align*}
  \text{Ad}_g &: G \longrightarrow G \, , \quad h \mapsto ghg^{-1} \\
  \text{Ad}_{g*} &: T_eG \longrightarrow T_eG \, , \quad A \mapsto \mathbf{g A g}^{-1}
\end{align*}

Here the notation $\mathbf{g}$ is used to denote the matrix product, since the adjoint action is defined on the group $G$ not the Lie algebra $\mathfrak{g}$.

Altogether the transition between two Yang-Mills fields on the intersection of two charts is given by:

\begin{align*}
\mathbf{\mathcal{A}}_j 
&= \mathbf{\Omega}^{-1} \, \mathbf{\mathcal{A}}_i \, \mathbf{\Omega} 
  + \mathbf{\Omega}^{-1} \,  d \mathbf{\Omega} \\
(\mathcal{A}_j)^i_{\; r \mu} 
&= \left( \Omega^{-1}(p) \right)^i_{\; k} \, (\mathcal{A}_i)^k_{\; l \mu} \, \Omega(p)^l_{\; r}
+ \left( \Omega^{-1}(p) \right)^i_{\; k} \, \partial_\mu \Omega(p)^k_{\; r}
\end{align*}

This is simply the \textbf{gauge transformation} as known form gauge theories\cite{NakaharaGeometrytopologyphysics2005}.

As an example, consider the case of a $U(1)$ principal bundle. The transition function \( \Omega \) is a smooth function \( U_i \cap U_j \longrightarrow U(1) \), which can be expressed as \( \Omega(m) = \exp[i\Lambda(m)] \) for some real-valued function \( \Lambda: U_i \cap U_j \longrightarrow \mathbb{R} \). Since $U(1)$ is a subgroup of $GL(d,\mathbb{C})$, two local connection forms \( \mathcal{A}_i \) and \( \mathcal{A}_j \) on the intersection \( U_i \cap U_j \) are then related by:

\begin{align*}
  \mathcal{A}_j 
  &= \Omega^{-1} \, \mathcal{A}_i \, \Omega + \Omega^{-1} \, d\Omega \\
  &= \mathcal{A}_i + e^{-i\Lambda(m)} \, d\left( e^{i\Lambda(m)} \right) \\
  &= \mathcal{A}_i + e^{-i\Lambda(m)} \cdot i e^{i\Lambda(m)} \, d\Lambda \\
  &= \mathcal{A}_i + i \, d\Lambda
\end{align*}

Which in components reads:
\[ \mathcal{A}_{j \, \mu} = \mathcal{A}_{i \, \mu} + i\partial_\mu \Lambda \]
This is the familiar form of the gauge transformation in electromagnetism\cite{NakaharaGeometrytopologyphysics2005}.

