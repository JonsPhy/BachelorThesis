
\section{Connections on Principal Bundles}

\subsection{General Definition}

Let \( P \xrightarrow{\pi} M \) be a principal bundle with structure group \( G \).
The right action of \( G \) on \(P \) induces a vector field as follows:
For each \( A \in \mathfrak{g} \cong T_eG \) the action of the one parameter subgroup \( \exp(tA) \) on \( p \in P \) yields a curve. Since the group acts within the fiber \( \pi(p) = \pi(p \triangleleft \exp(tA)) = p \). We define a vector \( X^A_p \in T_pP \) by its action on a function \( f \in C^\infty(P) \) \cite{NakaharaGeometrytopologyphysics2005}:
\[ X^A_p f = \frac{d}{dt} f(p \triangleleft \exp(tA))\mid_{t=0} \]

Futhermore a vector space isomorphism is defined \( i: \mathfrak{g} \longrightarrow \Gamma(TP) \) that assigns to each element \( A \in \mathfrak{g} \) the vector field \( X^A \). This is called a \textbf{fundamental vector field} on \( P \).

We define the \textbf{Pushforward} \cite{Pushforward2025} of a smooth map \( F:M \longrightarrow N \) of smooth manifolds \( M \) and \( N \) as the a map between the tangent spaces:
\[ F_* : T_pM \longrightarrow T_{F(p)}N \]

By identifying \( (F_*v)(f) = v(f\circ F) \) for \( v \in T_pM \) and \( f \in C^\infty(N) \).

The pushforward of the projection map \( \pi_* : TP \longrightarrow TM \) allows the construction of the \textbf{vertical subspace} \( V_pP \coloneq \ker(\pi_*) \) at a point \( p \in P \), which is a vector subspace of the tangent space of \( P \).

Notice that that each fundamental vector \( X^A_p \in V_pP \), since by construction \( \pi_* (X^A_p) = 0 \).


A \textbf{connection} on a principal bundle \( P \xrightarrow{\pi} M \) is a separation of the tangent space \( T_pP \) into a vertical subspace \( V_pP \) and a \textbf{horizontal subspace} \( H_pP \), by choosing a complement to the vertical subspace at each point \( p \in P \) such that:

\begin{align*}
  \text{(i)}\quad 
    & T_pP = H_pP \oplus V_pP \\
  \text{(ii)}\quad 
    & (\triangleleft g)_* (H_pP) = H_{p \triangleleft g}P 
    \quad \text{for all } g \in G \\
  \text{(iii)}\quad 
    & \text{For every smooth vector field } X \in \Gamma(TP), \text{ the unique decomposition } 
      X = X^H + X^V \\
    & \text{with } X^H(p) \in H_pP \text{ and } X^V(p) \in V_pP \text{ yields smooth vector fields } 
      X^H \in \Gamma(HP),\ X^V \in \Gamma(VP).
\end{align*}

The condition (ii) ensures that when moving along the fibers by the action of \( G \) the horizontal subspace changes in a smooth way, while (iii) guarantees that moving along \( P \) the horizontal subspace changes smoothly.


\subsection{Connection one-form}

The choice of a horizontal subspace at each point \( p \in P \) can be achieved by defining a Lie algebra valued one-form. The horizontal subspace is then interpreted as the kernel of this one-form. We define the \textbf{connection one-form} \( \omega \in \mathfrak{g} \otimes T^*P \) as a \( \mathfrak{g} \)-valued one-form on \( P \) such that:

\begin{align*}
  \text{(i)}\quad 
    & \omega(X^A) = A \quad \text{for all } A \in \mathfrak{g} \\
  \text{(ii)}\quad 
    & (\triangleleft g)^* \omega = \text{Ad}_{g^{-1}*} \omega \quad \text{for all } g \in G 
\end{align*}

Here, \( (\triangleleft g)^* \omega \) denotes the pullback
\footnote{The \textbf{pullback} of a of a one-form \( \omega \in \Gamma(T^*N) \) by a smooth map \( \phi: M \rightarrow N \) between smoth manifolds is defined as \( \phi^* \omega_p(X) = \omega_{\phi(p)}(\phi_{p*}(X)) \) for \( X \in T_pM \).\cite{Pullbackdifferentialgeometry2024}}
of the connection one-form by the right action of \( g \in G \) on \( P \), and \( \text{Ad}_{g^{-1}} \) is the adjoint action of \( g^{-1} \) on the Lie Group \( G \). That is \( \omega_{p \triangleleft g}(X_p(\triangleleft g)_*) = g^{-1} \cdot \omega_p(X_p) \cdot g \)


It will be stated without proof that any connection one-form satisfying these properties induces a horizontal subspace \( H_pP \) that satisfies the conditions of a connection \cite{FredericSchullerConncectionsconnection1formsLec21FredericSchuller2015}.



\subsection{local connection form}

The connection one-form as defined above is a global object on the principal bundle \( P \). However, in practice it is often useful to work with local connection forms. As will be shown in the next chapter, this local connection form is identified with the gauge potential in physical gauge theories.

Consider an open covering \( \{U_i\} \) of the base manifold \( M \) and local sections \( \sigma_i : U_i \rightarrow \pi^{-1}(U_i) \). We define a Lie-algebra valued one-form \[ \mathcal{A}_i \equiv \sigma_i^* \omega \in \mathfrak{g} \otimes\Omega^1(U_i) \] for a global connection one-form \( \omega \) \cite{NakaharaGeometrytopologyphysics2005}. Such a local connection form is called a \textbf{Yang-Mills field} \cite{FredericSchullerLocalrepresentationsconnectionbasemanifoldYangMillsfieldsLec222015}

Given a local section \( \sigma_i : U_i \to P \), one obtains a local trivialization:
\begin{align*}
  \psi_i : U_i \times G &\longrightarrow \pi^{-1}(U_i) \subset P \\
  (p, g) &\mapsto \sigma_i(p) \triangleleft g
\end{align*}


This trivialization introduces a local representation of the global connection one-form \( \omega \) via its pullback:
\begin{align*}
  \psi_i^* \omega : T_{(p, g)}(U_i \times G) &\longrightarrow \mathfrak{g} \\
  (\psi_i^* \omega)_{(p, g)}(X) &= \omega_{\sigma_i(p) \triangleleft g}\left((\psi_i)_* X\right)
\end{align*}


This local representation is related to the Yang-Mills field \( \mathcal{A}_i \) by\cite{FredericSchullerLocalrepresentationsconnectionbasemanifoldYangMillsfieldsLec222015}:

\begin{align*}
  (\psi_i^* \omega)_{(p, g)}(X) &= \text{Ad}_{g^{-1}*} \left(\mathcal{A}_i (X)\right) + \Xi_g(X) \\
\end{align*}

The above-used $\Xi$ is the \textbf{Maurer–Cartan form} of the Lie group \( G \). This form takes a tangent vector \( v \in T_gG \) and maps it to the unique Lie algebra element (i.e., a tangent vector at the identity) that generates \( v \) via left translation:
\[
\Xi(v) = (g^{-1} \triangleright)_* v.
\]
This uses the fact that every tangent vector on \( G \) arises as the pushforward of a unique element of the Lie algebra \( \mathfrak{g} = T_eG \) under the left action, i.e., for every \( v \in T_gG \), there exists a unique \( X \in \mathfrak{g} \) such that
\[
v = (g \triangleright)_* X.
\]
Thus, \( \Xi \) identifies the tangent bundle \( TG \) with \( G \times \mathfrak{g} \) via left translation\cite{MaurerCartanform2025}.

\subsection{Connection on the Frame Bundle}

The Frame Bundle $LM$ is of particular interest, because many groups relevant in physics are subgroups of the general linear group \( GL(n, \mathbb{R}) \) or \( GL(n, \mathbb{C}) \). Therefore in the following a local connection form and the Maurer-Cartan form will be derived.

Any choice of a chart \( (U_i, x) \) on the base manifold \( M \) induces a section on the frame bundle \( LM \) by associating to each point \( m \in U_i \) the frame given by its coordinates. This section is denoted as:

\begin{align*}
  \sigma : U_i &\longrightarrow LM \\
  m &\mapsto \sigma_i(m) \coloneq \left( \left. \frac{\partial}{\partial x^1} \right|_m, \dots , \left. \frac{\partial}{\partial x^{\text{dim}M}} \right|_m  \right)
\end{align*}



Then the Yang-Mills field \( \mathcal{A}_i = \sigma_i^* \omega \) is a one-form on \( U_i \) with values in the Lie algebra $\mathfrak{gl}(\text{dim}M,\mathbb{R}) = \{ M \mid M \text{ is a } n\times n \text{ matrix with components } M^\alpha_{\,\beta}\in \mathbb{R} \}$. The Yang-Mills field can be expressed in its components as:
\[ (\mathcal{A}^i)^\alpha_{\,\,\beta\mu} \]

Where $\alpha, \,  \beta$ are labels for the Lie algebra components and $\mu$ is the index of the base manifold. 

The Maurer-Cartan form \( \Xi \) can be constructed as follows:

Let \( gl \subseteq GL(d,\mathbb{R}) \) be an open subset of the general linear group containing the identity. Coordinates are introduced by:
\begin{align*}
  x: gl &\longrightarrow \mathbb{R} \\
  g &\mapsto x(g)^a_{\,\,b} \coloneq g^a_{\,\,b}
\end{align*}

Consider a left-invariant vector field \( L^A \) generated by the Lie algebra element \( A \in \mathfrak{gl}(d, \mathbb{R}) \). Since it is a vector field on the group, it acts on the coordinate functions:

\begin{align*}
  \left( L^A x^a_{\,\,b} \right)_g &= x^a_{\,\,b} \frac{d}{dt} \left( g \cdot \exp(tA) \right) \bigg|_{t=0} \\
  &= \frac{d}{dt} \left( g^a_{\,\,c} \exp(tA)^c_{\,\,b} \right) \bigg|_{t=0} \\
  &= g^a_{\,\,c} A^c_{\,\,b}
\end{align*}


Therefore the components of the vector field are given by \( L^A_g = g^a_{\,\,b} \, A^b_{\,\,c} \, \frac{\partial}{\partial x^a_{\,\,c}} \)\cite{FredericSchullerLocalrepresentationsconnectionbasemanifoldYangMillsfieldsLec222015}

The Maurer-Cartan form $\Xi$ then is defined as the one-form that maps the left-invariant vector field \( L^A \) to the Lie algebra element \( A \):
\[ (\Xi_g)^a_{\,\,b} = (g^{-1})^a_{\,\,c}(dx^c_{\,\,b}) \]

It can be easily checked that this expression satisfies the properties of a Maurer-Cartan form:

\begin{align*}
  \Xi_g(L^A_g) 
  &= (g^{-1})^a_{\,c} \, (dx)^c_{\,b} \left( g^p_{\,r} \, A^r_{\,q} \, \frac{\partial}{\partial x^p_{\,q}} \right) \\
  &= (g^{-1})^a_{\,c} \, g^p_{\,r} \, A^r_{\,q} \left( (dx)^c_{\,b} \, \frac{\partial}{\partial x^p_{\,q}} \right) \\
  &= (g^{-1})^a_{\,c} \, g^p_{\,r} \, A^r_{\,q} \, \delta^c_{\,p} \, \delta^q_{\,b} \\
  &= (g^{-1})^a_{\,p} \, g^p_{\,r} \, A^r_{\,b} \\
  &= A^a_{\,b}
\end{align*}


\subsection{Compatibility condition for local connection forms}


It was stated before, that the local connection forms $\mathcal{A}_i$ relate to a unique global connection one-form $\omega$. For this to be true, the local connection forms must satisfy a compatibility condition. This condition is given by the requirement that the local connection forms on overlapping charts \( U_i \cap U_j \neq \emptyset\) are related by a gauge transformation\cite{NakaharaGeometrytopologyphysics2005}. Specifically, let $\sigma_i$ and $\sigma_j$ be sections respectively defining Yang-Mills fields \( \mathcal{A}_i \) and \( \mathcal{A}_j \) on the overlapping region \( U_i \cap U_j \). Introduce a gauge map
\[ \Omega : U_i \cap U_j \longrightarrow G \]
defined by the relation
\[ \sigma_j(m) = \sigma_i(m) \triangleleft \Omega(m) \quad \forall m \in U_i \cap U_j \]


Then the local connection forms are related as follows:

\[ \mathcal{A}_j = \text{Ad}_{\Omega^{-1}(m)*} \mathcal{A}_i + \Omega^*\Xi_m \]

In this will be shown for the case of the frame bundle $LM$. First, we calculate the latter expression. Notice that $\Omega^*\Xi_m$ is a map from the tangent space of the intersection on the base manifold \( U_i \cap U_j \) to the Lie algebra \( \mathfrak{gl}(d, \mathbb{R}) \). Therefore to find the explicit form it is calculated how this map acts on a vector in the tangent space:

\begin{align*}
  (\Omega^*\Xi)_p \left( \frac{\partial}{\partial x^\mu} \right)_p  
\end{align*}


